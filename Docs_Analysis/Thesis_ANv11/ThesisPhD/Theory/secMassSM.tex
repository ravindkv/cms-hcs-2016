
\section{Mass of particles}
\label{s:smMass}

The Lagrangian, given in Equation (\ref{eq:smLag}), is complete in the sense that it 
is invariant under the SU(3)$_C\times $SU(2)$_L\times$ U(1)$_Y$ group and it is 
renormalizable. However, except for the Higgs boson, the mass terms for
other particles, such as the fermion mass term 
$m\bar\psi \psi$, 
are not included in Equation (\ref{eq:smLag}). The 
mass term for the Higgs boson is included in the potential 
\begin{equation}
V(\varphi) = \left(\partial_\mu\varphi\right)^\dagger \left(\partial^\mu
\varphi\right) - \mu^2\varphi^\dagger\varphi - \lambda\left(
\varphi^\dagger\varphi\right)^2,
\label{eq:potPhi}
\end{equation}
where, 
\begin{equation}
\varphi = \frac{1}{\sqrt 2} \left(\begin{array}{c}\varphi_1 (x) + 
i\varphi_2 (x) \\ \varphi_3 (x) + i\varphi_4 (x) \end{array}\right),
\label{eq:higgsPhi}
\end{equation}

After substituting $\varphi$ from Equation (\ref{eq:higgsPhi}) to the potential of
Equation (\ref{eq:potPhi}), it can be easily seen that all the four Higgs fields 
have a mass term of type $\mu^2\varphi_i^2$ where i = 1,2, 3, 4, where $\mu^2 < 0$. This
lead to spontaneous symmetry breaking as discussed in the next section.
Note that $\mu$ is not the physical mass of the Higgs particle. 
The physical masses of 
all particles are generated by the spontaneously breaking of symmetry 
\cite{PhysRev.127.965} and the Higgs mechanism 
\cite{PhysRevLett.13.321,HIGGS1964132,Isildak:2013kfa,PhysRevLett.13.508,PhysRevLett.13.585,PhysRev.145.1156,PhysRev.155.1554} as discussed below. 
\begin{itemize}[leftmargin=*]
\item \textbf{Spontaneous symmetry breaking (SSB) and Higgs mechanism}: 
The minima of the potential 
$V(\varphi)$ correspond to $\varphi_1^2 + \varphi_2^2 + 
\varphi_3^2 + \varphi_4^2 = v^2$, where $v = \sqrt{-\mu^2/\lambda}$ 
is the vacuum expectation value (vev), $\mu^2 < 0$, and $\lambda > 
0$. Next we  
 expand the potential about the minima.  
The minima can be chosen at 
$\varphi_1 = \varphi_2 = \varphi_4 = 0, \varphi_3 = v$, and the
potential in the Lagrangian $\mathcal{L}_H$ is expanded about this
minimum. That is, the field given by Equation (\ref{eq:higgsPhi})
is re-written as 
\begin{equation}
\varphi = \frac{1}{\sqrt 2} \left(\begin{array}{c} \varphi_1(x)+i\varphi_2(x)
\\ v + h(x) + i\varphi_4(x) \end{array}\right),
\label{eq:higgsPhi2}
\end{equation}
It turns out that the fields $\varphi_1$, $\varphi_2$ and $\varphi_4$ are
not physical and get eliminated from the physical particle spectrum. We 
see this by reparametrizing this equation as,
\begin{equation}
\varphi = \frac{e^{i\tau_a\theta^a(x)/v}}{\sqrt{2}}
\left(\begin{array}{c} 0
\\ v+h(x) \end{array}\right),
\label{eq:higgsPhi3}
\end{equation}
where $\theta^a$ are three new fields. We can now eliminate these fields 
by a gauge transformation. The Higgs multiplet can now be expressed as
\begin{equation}
\varphi = \frac{1}{\sqrt{2}}
\left(\begin{array}{c} 0
\\ v+h(x) \end{array}\right),
\label{eq:higgsPhi4}
\end{equation}
The resulting gauge choice is called the unitary gauge. In this gauge, only the physical Higgs field appears in the Lagrangian and
all the other scalar fields get eliminated. After substituting $\varphi$ from 
Equation (\ref{eq:higgsPhi4}) in $\mathcal{L}_H$, mass terms appear for gauge 
bosons $B_\mu$, $W_1$, $W_2$, and $W_3$. Furthermore the Higgs
field acquires physical mass. The other three scalar degrees of freedom now
act as the longitudinal modes of the massive gauge bosons. 

The $\mathcal{L}_H$ now contains only gauge fields $B_\mu, W_\mu^a$ and
the $h(x)$ with respective mass terms. We next identify the physical
gauge boson particle spectrum. We note that the vacuum preserves
the electromagnetic 
 $U(1)_{em}$ symmetry since 
 $\varphi_3$ has zero electric charge ($T^3 = -1/2$, 
$Y = 2$, hence $Q = 0$) hence does not transform under the 
$U(1)_{em}$ group. The gauge
bosons  are expressed in a new basis as 
$W^\pm = (W^1 \mp iW^2)/ \sqrt{2}, ~A_\mu = \cos\theta_W B_\mu 
+ \sin\theta_W W^3_\mu,
~Z_\mu = -\sin\theta_W B_\mu + \cos\theta_W W^3_\mu$. After writing
$\mathcal{L}_H$ in terms of $A_\mu, ~Z_\mu, ~W^\pm$, one can see that
the $A_\mu$ becomes massless and $W^\pm$, $Z_\mu$ remain
massive with masses $ \frac{1}{2}vg, ~\frac{1}{2}v\sqrt{g^2 + 
g^{\prime}}$, respectively. The mass of leptons (electron, muon, 
and tau) and quarks are generated using Equation (\ref{eq:higgsPhi3})
in $\mathcal{L}_{\rm{Yukawa}}$. The neutrino in the SM remains massless. 
\end{itemize}

Notice that before the symmetry breaking, there were 4 massless (gauge bosons) and 4 
scalar particles. The polarization degree of freedom ($P_{\text{dof}}$) 
for each Higgs is 1 and for each massless gauge boson is 2. 
Therefore, total $P_{\text{dof}}$ before the SSB is $4\times1 + 4\times 
2 = 12$. After the SSB, the massive gauge bosons get an additional $P_{\text{dof}}$
from the three scalar boson. Therefore, the total number of $P_{\text{dof}}$ after the SSB 
is also 12 ($1\times 1 + 3\times 0 + 1\times 2 + 3\times 3$). 

The masses of all the fundamental particles are shown in 
Table~\ref{tab:massSM} in terms of free parameters. However, the value of these 
parameters is determined from experiment.
Until now, the Lagrangian $\mathcal{L}_{\rm{SM}}$ contains 14 free parameters which 
are 3 couplings of gauge groups ($g, ~g^\prime, ~g_s$), 2 from Higgs field 
($v, ~\lambda$), 3 couplings ($G_l$) of Higgs-lepton interactions, and 6 couplings
($G_q$) of Higgs-quark interactions. There are 5 additional free parameters in the
SM which are 3 CKM mixing angles ($\theta_{12}, ~\theta_{23}, ~\theta_{13}$) 
\cite{PhysRevLett.10.531}, 1 CKM CP violation phase \cite{10.1143/PTP.49.652}, 
and 1 QCD vacuum angle ($\theta_{\rm{QCD}}$). Therefore, there are 19 free parameters 
in the SM.

\begin{table} 
\caption{The mass of all fundamental particles in terms of the free parameters of
				the standard model with the corresponding observed values.} 
\label{tab:massSM} 
\begin{centering} 
\begin{tabular}{ccc} 
\hline 
\hline 
Particles & Theoretical mass & Experimental mass \\ 
\hline 
\hline
\noalign{\vskip 0.1cm}
				H & $v\sqrt{2\lambda}$ & 125.1 \GeV \\ \noalign{\vskip 0.1cm}
				$W^\pm $ & $\frac{1}{2}vg$ & 80.4 \GeV \\ \noalign{\vskip 0.1cm}
				Z & $\frac{1}{2}v\sqrt{g^2 + g^{\prime}}$ & 91.2 \GeV \\ \noalign{\vskip 0.1cm}
				photon & 0 & 0\\ \noalign{\vskip 0.1cm}
				gluon & 0 & 0\\ \noalign{\vskip 0.1cm}
				$e, ~\mu, ~\tau$ & $\frac{G_lv}{\sqrt{2}}$ & shown in 
				Figure~\ref{fig:sm_particles}\\ \noalign{\vskip 0.1cm}
				neutrino & 0 & shown in Figure~\ref{fig:sm_particles}\\ \noalign{\vskip 0.1cm}
				Quarks & $\frac{G_qv}{\sqrt{2}}$  & shown in Figure~\ref{fig:sm_particles}\\
				\noalign{\vskip 0.1cm}
\hline 
\end{tabular} 
\par\end{centering} 
\end{table}

\section{Success and failure of the SM}
\label{s:meritSM}
The standard model is a mathematically complete and experimentally tested theory.
However, it does not provide explanations to all of the observed phenomena in
nature. A brief description of successes and failures of the SM is given below.
\begin{itemize}[leftmargin=*]
				\item \textbf{Success}: The SM describes the interaction among the 
								fundamental particles with excellent accuracy. The electro-weak
								gauge bosons ($W^\pm, ~Z$), gluon, top, and charm quarks were
								predicted by the SM before being observed experimentally. The
								Higgs boson was predicted in 1964 and experimentally observed in
								2012 \cite{Aad:2015zhl}. Value to all the free parameters of the 
								SM has been experimentally assigned. The predictions from the SM 
								agree very well with the experimental observations.

\item \textbf{Failure}:
There are many shortcomings of the SM. A few of them are mentioned below:  
\begin{itemize}[leftmargin=*]
\item The $\mathcal{L}_{SM}$ includes 
three generations of leptons. However, only the first generation ($e, \nu_e$) is 
enough to form the nucleus and hence the visible content of the universe. The SM
does not provide any explanation for the existence of the second and the third
generations or why there should not be more than three?
\item The neutrinos in the SM are massless. However, it is observed by 
the SNO experiment that neutrinos do have a small mass. 
\item Third, why the Lagrangian $\mathcal{L}_{\rm{SM}}$ does not 
include gravity?
\item The SM does not explain the origin of dark matter \cite{Ade:2013zuv,
Ade:2013sjv} and dark energy which account for about 96\% of the universe. 
\item Why there are so many (19) free parameters in the SM? 
\item Why there is more matter than antimatter, etc? 
\end{itemize}
\end{itemize}
To explain the shortcomings of the SM, many beyond the standard model (BSM) 
theories have been proposed such as grand unified theories, supersymmetry 
\cite{Martin:1997ns}, etc. One of the extensions of the SM that predicts the 
existence of charged Higgs boson is described in the next chapter.
