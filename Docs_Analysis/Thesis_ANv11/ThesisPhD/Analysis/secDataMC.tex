The physics objects of our interest are primary and secondary vertices, 
lepton, jets, and missing transverse energy attributed to neutrino. The particle flow (PF) 
algorithm~\cite{CMS-PAS-PFT-09-001,CMS-PAS-PFT-10-001} is used to 
reconstruct these objects. In this chapter, we first list the collision data 
and simulated Monte Carlo (MC) samples used for this analysis and then describe the 
reconstruction process for the primary and secondary vertices, muon, electron, jets, and 
\PQb tagging.

%%%%%%%%%%%%%%%%%%%%%%%%%%%%%%%%%%%%%%%%%%%%%%%%%%%%%%%%
\section{Data samples}
The reconstructed data is broadly divided into different event topologies. 
Only those datasets matching with event topology under consideration are used 
for this analysis. Such approach help in optimising the computing resources.
The single muon and single electron data sample used for this analysis are 
shown in Table~\ref{tab:dataSample}. The data was collected in 2016. 
A luminosity mask is applied to select good quality data, where most of the 
sub-detectors of the CMS operates in a normal condition. The integrated luminosity
after applying the mask reduces marginally from 37.0\fbinv to 35.9\fbinv.

\begin{table}
\begin{center}
\caption{Muon and electron data samples recorded in 2016 with an integrated 
	luminosity 35.9\fbinv.} 
\label{tab:dataSample}
\begin{tabular}{ccc} \hline\hline                                                         {\bf{Dataset}} & {\bf{$L_{\rm{int}} (\fbinv)$}} & {\bf{Events}} \\\hline\hline
\verb|SingleMuon|    & 35.86 & 786767595  \\[0.1cm]
\verb|SingleElectron|& 35.86 & 937954379 \\\hline
 \end{tabular}          
\end{center}
\end{table}

\section{Simulated samples}
The centrally produced official simulated samples are used in this analysis. 
All the simulated samples  are shown in Table~\ref{tab:mcSample} with 
corresponding cross section and number of events. The k-factor is the 
ratio of cross sections from next-to-leading order and leading order, 
k$_\rm{f} =\sigma_{\text{NLO}}/\sigma_{\text{LO}}$.

The signal and background simulated samples are generated using \MGvATNLO~\cite{Alwall:2011uj, Alwall:2014hca}
as well as \POWHEG~\cite{Frixione:2007vw, Nason:2004rx, Alioli:2010xd} generators at parton level. 
These parton level events are hadronised using \PYTHIA~\cite{Sjostrand:2006za, Sjostrand:2007gs}. 
The hadronised events were tuned using  CUETP8M1~\cite{CMS-PAS-TOP-16-021} which are then passed
through \GEANTfour~\cite{Agostinelli:2002hh} for detector simulation. Finally, the events are
reconstructed after complete detector simulation. 

The \ttjets channel is the most significant irreducible background for this analysis. It
contributes around 94\% of the total backgrounds in the signal region. The parton level events 
for \ttjets samples of Table~\ref{tab:mcSample} are generated using 
\POWHEG~\cite{Frixione:2007vw, Nason:2004rx, Alioli:2010xd} at next-to-leading (NLO) order. The 
\verb|NNPDF30_nlo_as_0118| ~\cite{Ball:2014uwa} parton distribution function (PDF) was used 
for this purpose. The partonic events were then hadronised using 
\PYTHIA~\cite{Sjostrand:2006za, Sjostrand:2007gs}. To simulate this channel at higher orders and 
to take care of non-perturbative effects, \ttjets samples were tuned with 
CUETP8M1~\cite{CMS-PAS-TOP-16-021}. The NNLO cross section for \ttjets process is estimated to be 
$831.76 \pm^{20}_{29}(\text{scale}) \pm 35 (\text{PDF} + \alpha_s)$ pb \cite{Beneke:2011mq}. 

Single \PQt quark samples are shown in Table~\ref{tab:mcSample}, where a \PQt quark is produced with 
jets in \PQt-channel (\verb|ST_t|), \PQs-channel (\verb|ST_s|), and \PQt\PW-channel (\verb|ST_tW|). The
\verb|ST_tW| and \verb|ST_t| samples are produced using \POWHEG + \PYTHIA and CUETP8M1, where \verb|ST_tW| has 
5-flavor scheme whereas \verb|ST_t|, has 4-flavor scheme and further, the \PQt quark decays 
inclusively. The \verb|ST_s| are generated using \MGvATNLO~\cite{Alwall:2014hca} in the 4-flavor 
scheme and was hadronised using \PYTHIA. The NLO cross section for single \PQt process 
\cite{Aliev:2010zk, Kant:2014oha} are shown in Table~\ref{tab:mcSample}.

The inclusive \wjets and \dyjets samples were generated using \MGvATNLO and are hadronised using 
\PYTHIA. The MLM~\cite{ Alwall:2007fs} technique was used to take care of double counting of partons. There
are exclusive \PW + n jets and $\PZ/\PGg$ + n jets samples available for n = 1, 2, 3, 4. 
The inclusive (\wjets and \dyjets) and exclusive ($\PW/\PZ/\PGg$ + n jets) samples have our event 
topology \ie 4 jets + 1 lepton + \MET. Therefore, the inclusive and exclusive samples are added 
appropriately to have more statistics. To avoid double counting, the exclusive samples are normalized
w.r.t the inclusive one. For example, the $\PW$ + 1 jet events are multiplied by 
$1/(L_{\wjets} + L_{\PW + \text{1 jet}})$ and then are added linearly with that of \wjets. 
Where $L_{\wjets}$ is the luminosity (number of events divided by the cross section) of \wjets sample. A similar procedure is followed to add $\PZ/\PGg$ + n jets with \dyjets. The NLO cross section for 
these processes is shown in Table~\ref{tab:mcSample}.

The vector boson pair production process (WW, WZ, and ZZ) samples are generated and hadronised 
using \PYTHIA and are tuned using CUETP8M1. The NLO cross section for these samples are shown 
in Table~\ref{tab:mcSample}.

The muon (electron) enriched QCD multijet samples are used for \mujets (\ejets) channel. These 
samples are generated using \PYTHIA and CUETP8M1. After multiplying with filter efficiency, the 
NLO cross sections are shown in Table~\ref{tab:mcSample}. From this table, it can be seen that the 
luminosity for these samples are significantly 
smaller as compared to the observed luminosity and therefore, a data-driven approach is used to 
make more precise estimation of QCD multijet background.

The charged Higgs signal samples were generated using \MGvATNLO, and hadronized using \PYTHIA. The 
signal sample for several mass points in the range of 80 to 160 \GeV (80, 90, 100, 120, 140, 150, 
155, 160) are generated for the search for charged Higgs. The cross section 
for the signal is 831.76$\times$ 0.2132 pb, where 831.76 \unit{pb} is the inclusive 
(fully-hadronic, fully-leptonic, semi-leptonic) \ttjets production cross section and factor 
0.2132 is the branching fraction of $W^-\to l^- \bar{\nu_l}$ (where $l = \mu, e$, $\tau$ is not 
considered in this analysis) \cite{Beringer:1900zz}. The factor 0.2132 is multiplied because \ttbar decays 
semi-leptonically ($(t\to H^+ b, H^+\to c\bar{s}), (\bar{t}\to W^-\bar{b}, W^- \to l^-\bar{\nu_l})$), 
where $l = \mu, e$ in the charged Higgs signal samples. Furthermore, the signal events are scaled by 
the maximum observed upper limit obtained at 8 \TeV. The upper observed limit on \brThb is 6.5\% for
the $m_{H+}$ = 90 \GeV\cite{Khachatryan:2015uua}. Therefore, the signal samples are scaled by a factor 
$2\times 0.065\times (1-0.065) = 0.12155$ in every plot and table except in the data cards used for 
limit computation. 

\begin{table}
\caption{Background and signal simulated samples with corresponding cross section and the number of events.}
\label{tab:mcSample}
\begin{center}
\begin{tabular}{cccc} \hline\hline
    {\bf{Process}} & {\bf{$\sigma$}}  (pb) (Order) &{\bf{$k_\rm{f}$}}& {\bf{Events}}\\\hline
\hline
    \ttjets &$831.76^{+20}_{-29} \pm 35$ (NNLO) & - & 77081156 \\  
    \verb|Single t (tW-channel)| &$71.7 \pm 1.80 \pm 3.40$ (NLO) & - &  6933094   \\
    \verb|Single t (t-channel)|  &$80.95^{+5.8}_{-5.2} \pm 0.16$ (NLO) & -  & 38811017  \\ 
    \verb|Single t (s-channel)|  &$10.32^{+0.6}_{0.5} \pm 0.01$ (NLO) & -  & 2989199  \\ \hline
    \verb|W + jets|   &$50690 \pm 389.1$ (LO) & 1.21  & 29181900  \\
    \verb|W + 1 jet|  &$9493 \pm 25.52$ (LO)      & 1.21  & 44813600  \\
    \verb|W + 2 jets| &$3120 \pm 78.5$ (LO)   & 1.21  & 29878415  \\
    \verb|W + 3 jets| &$942.3 \pm 36.8$ (LO)  & 1.21  & 19798117  \\
    \verb|W + 4 jets| &$524.2 \pm 23.6$ (LO)  & 1.21  & 9170576  \\\hline

    \verb|Drell-Yan + jets|  &$4895 \pm 41$ (LO)    & 1.17  & 48103700  \\
    \verb|Drell-Yan + 1 jet| &$1016 \pm 16.8$ (LO)  & 1.17  & 62079400  \\
    \verb|Drell-Yan + 2 jets|&$331.3\pm 8.5$ (LO)   & 1.17  & 19970551  \\
    \verb|Drell-Yan + 3 jets|&$96.6 \pm 3.9$ (LO)   & 1.17  & 5856110  \\
    \verb|Drell-Yan + 4 jets|&$51.4 \pm 2.5$ (LO)   & 1.17  & 4197868  \\ \hline

    \verb|WW|  &$118.7^{+2.5}_{-2.2}$ (NNLO) & -   & 994012  \\ 
    \verb|WZ|  &$46.74^{+1.9}_{1.5}$ (NLO)  & -  & 1000000 \\ 
    \verb|ZZ|  &$17.72^{0.6}_{0.4}$ (NLO)  & -  & 990064  \\ \hline
    \verb|QCD_Pt-15to20_MuEnriched|    & 3819570 (NLO) & -  & 4141251   \\
    \verb|QCD_Pt-20to30_MuEnriched|    & 2960198 (NLO) & -  & 31475157  \\
    \verb|QCD_Pt-30to50_MuEnriched|    & 1652471 (NLO) & -  & 29954815  \\
    \verb|QCD_Pt-50to80_MuEnriched|    & 437504  (NLO) & -  & 19806915  \\
    \verb|QCD_Pt-80to120_MuEnriched|   & 106033  (NLO) & -  & 13786971  \\
    \verb|QCD_Pt-120to170_MuEnriched|  & 25190   (NLO) & -  & 8042721   \\
    \verb|QCD_Pt-170to300_MuEnriched|  & 8654    (NLO) & -  & 7947159   \\
    \verb|QCD_Pt-300to470_MuEnriched|  & 797     (NLO) & -  & 7937590   \\\hline
    
    \verb|QCD_Pt-15to20_EMEnriched|      & 254600  (NLO) & -  & 5652601    \\
    \verb|QCD_Pt-20to30_EMEnriched|      & 5352960 (NLO) & -  & 9218954    \\
    \verb|QCD_Pt-30to50_EMEnriched|      & 9928000 (NLO) & -  & 4730195    \\
    \verb|QCD_Pt-50to80_EMEnriched|      & 2890800 (NLO) & -  & 22337070   \\
    \verb|QCD_Pt-80to120_EMEnriched|     & 350000  (NLO) & -  & 35841783   \\
    \verb|QCD_Pt-120to170_EMEnriched|    & 62964   (NLO) & -  & 35817281   \\
    \verb|QCD_Pt-170to300_EMEnriched|    & 18810   (NLO) & -  & 11540163   \\
    \verb|QCD_Pt-300toInf_EMEnriched|    & 1350    (NLO) & -  & 7373633    \\\hline
    \verb|ChargedHiggsToCS_M080_To160|  & $0.2132 \times 831.76$ (NNLO) & - &1000000   
    \\\hline
\end{tabular}
\end{center}
\end{table}

In the \ttjets samples, the NLO matrix element parton shower matching is varied by the damping 
parameter ($h_\text{damp}$). Additional $\ttjets$ samples are generated by varying 
$h_\text{damp}$ up and down and are used to observe the effect of $h_\text{damp}$. Similarly, 
$\ttjets$ samples where the renormalization and factorisation scales have been varied up 
and down are used to evaluate the uncertainties due to these scales. Also, alternate \ttjets 
samples with \mt = 171.5 and 173.5\GeV are considered to observe the effect of mass of the \PQt quark. 
All the additional \ttjets samples used for systematics study are shown in Table~\ref{tab:mcSampleSys}. 
\begin{table}
\caption{ The SM \ttjets samples used for systematics study.}
\label{tab:mcSampleSys}
\begin{center}
\begin{tabular}{ccc} \hline\hline
    {\bf{Process}} & {\bf{$\sigma$}} (pb) & {\bf{Events}}\\\hline\hline
	\ttjets , scale up           		&  730 & 29310620 \\
	\ttjets , scale down         		&  730 & 28354188 \\
	\ttjets , \mt = 173.5 \GeV           &  711 & 19419050 \\
	\ttjets , \mt = 171.5 \GeV           &  750 & 19578812 \\
	\ttjets , $h_\text{damp}$ up     	&  750 & 29689380 \\
	\ttjets , $h_\text{damp}$ down    	&  750 & 29117820\\\hline
\end{tabular}
\end{center}
\end{table}
