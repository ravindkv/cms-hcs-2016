A series of filters~\cite{metFilters} are applied to
filter \dq{bad} events, such as those where there is a huge difference in the \pt of
muon measured from the tracker and the muon chambers. Although the occurrence of such
\dq{bad} events is rare, these filters are applied as sanity measures. These
filters mostly affect the actual data, most of them have no effect on the simulated
events. The efficiency of all combined filters is 99.4\% for data, and 99.9\%
for simulated \ttjets sample. These filters are particularly useful in those
analyses where there is a requirement of very high \MET and the final event yield is 
small. The list of filters with their availability and applicability is shown
in Table~\ref{tab:eventFilters}. A detailed description of each filter is given below:
\begin{itemize}[leftmargin=*]
	\item \textbf{HBHE noise filter}: It is known that the barrel and endcap
		part of the HCAL records noise (sporadic anomalous signals) at a
		fixed rate even if there is no stable beam for physics data taking
        \cite{metFilters}. Such an event is filtered.
	\item \textbf{HBHE isolated noise filter}: This filter first finds the
		candidates for rechits and rejects the event if the sum of
		iso-tagged energy is $>$ 50 \GeV, the sum of iso-tagged transverse
		energy $>$ 25 \GeV, and the number of iso-tagged rechits is $>$ 9.
 	\item \textbf{CSC beam halo filter}: The actual physics objects of interest
		are those which are produced in the proton-proton collisions. However, there
		are machine induced particles, produced through beam-gas or
		beam-pipe interactions, which fly with the beam at a large radius
		(up to 5\unit{m}). These machine induced particles are reconstructed mainly
		in the cathode strip chambers (CSC) as muon candidate. An event is
		filtered if such particles are present.
	\item \textbf{Primary vertex filter}: The events are required to have
		at least one good primary vertex. Therefore, those events are
		filtered where there is no good vertex. A good vertex is
		required to be real, the z-component of the position is $ <$ 24\unit{cm},
		the $\rho$ component of the position is $<$ 2\unit{cm}, and the minimum
		number of degrees of freedom is $>$ 4.
	\item \textbf{Bad EE supercrystal filter}: The events are filtered if
		there is an unusually large energy deposit in the four superclusters
        of ECAL endcaps \cite{metFilters}. This filter is applied only on data.
	\item \textbf{ECAL dead cell trigger primitive filter}: If the energy of
		the ECAL crystals are estimated from the trigger primitive and
		lies near the saturation energy then that event is filtered
		\cite{metFilters}.
	\item \textbf{Bad PF muon filter}: An event is filtered if the muon is
		although qualified as a PF muon but the quality is very low, and
		the \pt is large. The events are filtered if a muon is found
		which has \pt $>$ 100 \GeV, segment compatibility $< 0.3$ or relative
		percentage error in the \pt of the best (inner) track is more than
		200\% (100\%), and $\Delta \rm R$ (muon, PF muon) $< 0.001$.
	\item \textbf{Bad charged hadron filter}: Those events are filtered where
		the quality of the muon is low and it is not declared as PF muon.
		However, this non-PF muon is used in the calculation of PF-\MET
		as a candidate for charged hadron. To reject such events, the muon
		is required to have same \pt, segment compatibility, and error in
		track cuts as that of the above filter. An additional requirement
    of $\Delta \rm R$ (muon, charge hadron) $<$ 0.00001, and \verb|PtDiffRell|
		$< 0.00001$ is applied where \verb|PtDiffRell| =
		(\pt of PF candidate - \pt of muon inner track)/
		(0.5$\times$ (\pt of PF candidate + \pt of muon inner track)).
\end{itemize}

\begin{table}
\caption{ List of event filters applied to data and simulated samples. Most of the filters
	are available in the MINIAOD dataset through the trigger collection. The
	unavailable filters are applied on the fly using EDFilters.}
\label{tab:eventFilters}
\centering
\begin{adjustbox}{max width=\textwidth}
 \begin{tabular}{ccc}\hline\hline
  {\bf{Name of the filter}} & {\bf{Available in MINIAOD}} & {\bf{Applied on}}\\\hline\hline
  \verb|HBHE noise filter|            & Yes & Data \& Simulation \\
  \verb|HBHE isolated noise filter|   & Yes & Data \& Simulation \\
  \verb|CSC beam halo filter|         & Yes & Data \& Simulation \\
  \verb|Good primary vertex filter|   & Yes & Data \& Simulation \\
  \verb|Bad EE supercrystal filter|   & Yes & Data  \\
  \verb|ECAL dead cell trigger primitive filter| & Yes & Data \& Simulation\\
  \verb|Bad PF Muon Filter|  & No & Data \& Simulation \\
  \verb|Bad Charged Hadron Filter|        & No & Data \& Simulation \\\hline
 \end{tabular}
\end{adjustbox}
\end{table}

