
\section{95\% CL upper limit on \brThb} 
\label{s:secLimit}
The event yields from different event categories with systematics and statistical
uncertainties are shown in Tables~\ref{tab:eventYieldInc}, \ref{tab:eventYieldCTagInc}, 
and \ref{tab:eventYieldCTagEx} for \mujets and \ejets channel. From these tables, it can 
be seen that the total background event matches well with the data within the uncertainties. 
From Figures~\ref{fig:mjjBTagKinFit}, \ref{fig:mjj_cTagL}, and \ref{fig:mjj_cTagEx} it can
be seen that no excess has been observed for the mass of charged Higgs in the range
80 to 160 \GeV. In view of this, an exclusion limit at 95\% CL~\cite{Junk:1999kv} 
using profile likelihood ratio has been put on the \brThb, assuming \brHcs = 100\%.
The absence or presence of charged Higgs in the data is characterized by the exclusion limits. 
Less is the value of limit, the more is the possibility of the absence of charged Higgs signal 
and vice-versa.

However, the procedure for setting the upper limit is not straight-forward as it involves many 
mathematical techniques and intensive computing resources as there are so many bins of the $\mjj$ 
distributions, statistical and systematics uncertainties, etc. A detailed mathematical description
of the limit setting procedure is given in \cite{Cowan:2010js}. A precise description of it can be 
found in Chapter 2 and 5 of \cite{ATLAS:2011tau}. Since there are good literature how the 95\%
CL limit is calculated, we do not discuss mathematical details about it in this thesis. The only thing
that is specific to this analysis is setting the limit on \brThb rather than on the signal strength.
For this, the signal and background yield that goes in to the likelihood is scaled appropriately.
For example, denoting x as the branching ratio of \PQt quark decaying to charged Higgs and a bottom 
quark
\begin{equation}
\rm{x} = \brThb ,
\end{equation}
the different yields are scaled as
\begin{equation}
x^2\rm{N}^{\rm{HH}} + 2x(1-x)\rm{N}^{\rm{HW}} + (1-x)^{2}\rm{N}^{\rm{WW}}_{\ttjets} + \rm{N}_{\rm{other bkg}} ,
\label{eq:deltaN}
\end{equation}
where we have assumed that \brThb + \brTwb = 1. The $\rm{N}^{\rm{HW}}$, $\rm{N}^{\rm{WW}}_{\ttjets}$, 
and $\rm{N}_{\rm{other bkg}}$ is the number of events from 
simulated signal, SM \ttjets, and other background process as shown in Table~\ref{tab:eventYieldCTagEx}
from exclusive categories for \mujets and \ejets channel.
The contribution of $\rm{N}^{\rm{HH}}$ is very small, hence this process is not considered in this
analysis. The Higgs combine tool package is used to compute asymptotic limit on \rm{x} using the 
CLs method~\cite{Junk:1999kv}. Since the signal search region lies in the range 80 to 160 \GeV of 
$\mjj$, a $20 \geq \mjj \geq 170$ cut is used to avoid low statistical bins for further 
analysis. The limits are computed for each event category described in Chapter~\ref{c:secMassReco}. 



%\section{95\% CL limits}
%In the absence of any excess in the data over background, an exclusion upper 
%limit is set on the branching ratio of top-quark decaying to the charged Higgs
%and a b-quark. The procedure for setting the upper limit is not straight-forward
%as it involves many mathematical techniques and intensive computing ressources due
%to so many bins of the $\mjj$ distributions, statistical and systematics 
%uncertainties, etc. In this thesis, we briefly discuss the mathematical details 
%of the procedure starting from a very simple counting experiment without 
%considering any uncertainty. After that, we will genralise the procedure for a 
%shape analysis (where there are multiple bins) and discuss how the statistical 
%and systematics uncertainties are incorporated. Finally, we will discuss how a 
%physics model is included to set the upper limit on a specific parameter of 
%interest.
%
%\subsection{For a counting experiment}
%A simple counting experiment is one where we have final event yields in the signal
%region after applying a set of cuts. Let us denote the yields by $N_{obs}$, 
%$N_{sig}$, and $N_{bkg}$ for data, simulated signal, and simulated background
%process, respectively. The starting point to compute the limit is to construct
%a liklehood function and then the test stastics which is the negative log of
%liklehood functions. There is a free parameter (denoted by $r$) in the liklehood 
%function which represents the signal strength. For example, $r$ = 0 implise the
%background-only hyposthesis ($H_{bkg}$) , that is, the observed data is consistent 
%with the background prediction. However, the non-zero value of $r$ corresponds to 
%the signal + background hyposthesis ($H_{sig+bkg}$) . The upper limit on $r$ is set
%based on these two hyposthesises. The liklehood function is constructed by asuming
%that the $N_{obs}$ followes a poisson distribution where the rate is given by 
%$N_{sig} + N_{bkg}$ for $H_{sig+bkg}$, and $N_{bkg}$ for $H_{bkg}$. For a 
%simple signal model where the signal yield is linearly scaled by $r$ and the 
%background yield is not scaled, we can rewrite the different event yields as
%followes
%\begin{equation}
%	N_{obs} = n, ~N_{sig} = rs, ~ N_{bkg} = b
%\end{equation}
%The liklehood function is given by 
%for GOF. The test statistics as a measure of GOF is given as \cite{Baker:1983tu}
%\begin{equation}
%q_{GoF, saturated} = -2\ln \left(\frac{L_{nominal} (n|\mu s(\theta) + b(\theta))}{L_{saturated}(n|n)}\right)
%\label{eq:gof}
%\end{equation}
%
%The likelihood functions of Eq.~\ref{eq:gof} are given by
%\begin{equation}
%L_{nominal} (n|\mu s(\theta) + b(\theta)) = \prod_{j=1}^N \frac{(\mu s_j(\theta) + b_j(\theta))^{n_j}}{n_{j}!} \exp(-(\mu s_j(\theta) + b_j(\theta)))
%\label{eq:lNominal}
%\end{equation}
%and,
%\begin{equation}
%L_{saturated} (n|n) = \prod_{j=1}^N \frac{(n_j)^{n_j}}{n_{j}!} \exp(-n_j)
%\label{eq:lSaturated}
%\end{equation}
%
%
%Where, $n_j$ is the expectation value in $j^{th}$ bin of data, N is the total number of bins. The
%$s_j(\theta)$ and $b_j(\theta)$ are mean number of events in $j^{th}$ bin of signal and background
%process and depend on the nuisance parameters ($\theta$). The $\mu$ is the signal strength which is
%zero for background only hypothesis and non-zero for signal+background hypothesis. In $j^{th}$ bin,
%$s_j(\theta)$ and $b_j(\theta)$ are given by \cite{Cowan:2010js}
%
%\begin{equation}
%s_{j}(\theta) = s_{tot}\int_{j^{th} bin} f_s(x; \theta_s)dx,
%\end{equation}
%
%\begin{equation}
%b_{j}(\theta) = b_{tot}\int_{j^{th} bin} f_b(x; \theta_b)dx.
%\end{equation}
%The $f_s(x; \theta_s)$ and $f_b(x; \theta_b)$ are PDFs for signal and background events. 
%The $s_{tot}$ and $b_{tot}$ are mean of total number of events from signal and background process.
%
%
%\subsection{For a shape analysis}
%\subsection{Incorporating uncertainties}
%\subsection{Combining different channels}
%\subsection{Incorporating a physics model}
