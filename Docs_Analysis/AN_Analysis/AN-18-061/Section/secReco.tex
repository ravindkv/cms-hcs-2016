The physics objects of our interest are primary and secondary vertex, lepton and jets, missing transverse energy and particle/jet id. 
The Particle Flow (PF) algorithm~\cite{CMS-PAS-PFT-09-001,CMS-PAS-PFT-10-001} is used to reconstruct these objects.
Four-momentum of final state objects is tuned by applying top mass constraint using kinematic
fitting package. This helps in improving the dijet mass resolution.
Reconstruction of primary and secondary vertex, physics objects, b-tagging, c-tagging, and kinematic fitting are described in the following sections.

%--------------------------------------
% Primary-vertex reconstruction
%--------------------------------------
\subsection{Primary Vertex and Diffuse Offset Energy Density}
\subsubsection{Primary Vertex}
\label{s:secPV}
    To reconstruct a primary vertex (PV) following three steps are performed~\cite{chatrchyan:2014fea}:
    \begin{itemize}
        \item {\bf{ track selection}}: Primary vertices are associated with a large number of tracks
            emerging from the same point. Tracks satisfying following criteria are used to reconstruct the collision point (vertex). 
           \begin{itemize}
               \item distance of closest approach from the center of beam-spot $< 2.4$ cm,
               \item number of the associated pixel hits $\geq 2$, and
               \item number of the associated pixel+strips hits $\geq 5$.
           \end{itemize}
      \item {\bf{ track clustering}}: After track selection, the level of association of tracks from a vertex is quantified.
          This procedure is called track clustering where tracks are clustered according to their $z$-position
          from the center of beam-spot. The Deterministic Annealing algorithm~\cite{an-11-014, 726788} is used for such
          clustering. After clustering, the candidate vertices are found along with the tracks associated with them.
      \item {\bf{ vertex-position fitting}}: The $z$-position of candidate vertices with at least two tracks
           are fitted using the {\em adaptive vertex fitter}~\cite{Fruhwirth:1027031}. The fit returns
           $x$, $y$, $z$-position, and covariance matrix of the PV along with other parameters such as the number
           of degrees of freedom ($n_{\rm dof}$) and a weight associated with each track. We require $n_{\rm dof} > 4$.
   \end{itemize}

The tag used to access PV information from {\em MINIAOD} samples is: {\em offlineSlimmedPrimaryVertices}.
The pileup distribution in data is calculated with {\em pileupCalc} tool using the JSON file:
\verb|Cert_271036-284044_13TeV_23Sep2016ReReco_Collisions16_JSON.txt|, with a minimum bias cross-section of 69.2 mb. 
The MC samples listed in Table~\ref{tab:mcSample} were generated before the actual data taking in 2016.
While simulating these samples, the pileup distribution as initially seen in the data was used. 
During the actual data taking, however, the pileup distributions were changed.
To have a similar pileup distribution as in data, the MC events are re-weighted.
The weights applied on MC are shown in Figure~\ref{subfig:ratioPileup_all}.
After applying the pileup weights, the distribution of number of primary vertices per event (nvtx) is shown in
Figures~\ref{fig:nvtxRho}a and \ref{fig:nvtxRho}b for the \mujets and \ejets channel, respectively.

\subsubsection{Diffuse Offset Energy Density}
From Figures~\ref{subfig:nvtx_muKinFit} and \ref{subfig:nvtx_eleKinFit} it can be seen that there is a poor
agreement between data and simulations even after applying the pileup reweighting.
A similar trend is seen in many analyses at 13 TeV~\cite{CMS-AN-17-003, CMS-AN-17-116}. There is 
another variable similar to nvtx called diffuse offset energy density ($\rho$)~\cite{Cacciari:2005hq, Cacciari:2011ma}.
For $n$ number of jets in an event, $\rho$ is given by~\cite{Cacciari:2007fd}
\begin{equation}
    \rho = {\rm median}\left(\frac{p_{T, n}}{A_n}\right)
    \label{eq:rho}
\end{equation}
where $p_{T,n}$ and $A_n$ are the transverse momentum and area of $n$-th jet.
The tag {\em fixedGridRhoAll} is used to get $\rho$ from the {\em MINIAOD} files.
The $\rho$ distribution is shown in Figures~\ref{subfig:rhoAll_muKinFit} and \ref{subfig:rhoAll_eleKinFit}. 
It can be seen that $\rho$ has a better agreement between data and simulations compared to the nvtx
distribution. The $\rho$ is used in the calculation of isolation variable for muon and 
electron as shown in Equation (\ref{eq:ele_iso}). The jet energy corrections are also 
derived using $\rho$ as discussed in Section (\ref{ss:jet_id}).

\begin{figure}
    \centering  
    \subfigure[nvtx distribution \label{subfig:nvtx_muKinFit}]
    {\includegraphics[width=0.49\linewidth]{Image/Muon/KinFit/nvtx_muKinFit.pdf}}
    \subfigure[nvtx distribution \label{subfig:nvtx_eleKinFit} ]
    {\includegraphics[width=0.49\linewidth]{Image/Electron/KinFit/nvtx_eleKinFit.pdf}}
    \vfil
    \subfigure[$\rho$ distribution \label{subfig:rhoAll_muKinFit}]
    {\includegraphics[width=0.49\linewidth]{Image/Muon/KinFit/rhoAll_muKinFit.pdf}}
    \subfigure[$\rho$ distribution \label{subfig:rhoAll_eleKinFit}]
    {\includegraphics[width=0.49\linewidth]{Image/Electron/KinFit/rhoAll_eleKinFit.pdf}}
    \caption{Number of primary vertices (nvtx) and diffuse offset energy density ($\rho$) distributions 
        for \mujets and \ejets channels. The data-MC agreement is better for $\rho$ compared 
    to the nvtx distribution. For lower values of $\rho$, there is a significant mismatch between data 
    and MC. Similar trend is seen in other analysis 
    as well~\cite{CMS-AN-17-003, CMS-AN-17-116}.}
    \label{fig:nvtxRho}
\end{figure}

%--------------------------------------
% Muon Reconstruction
%--------------------------------------
\subsection{Muon}
\subsubsection{Reconstruction}
Muons are often referred to as minimum ionising particles. Their energy loss is much smaller than 
electrons due to its larger mass. Muon in CMS detector with sufficiently higher \pt can traverse through most of the sub-detectors such as tracker calorimeter and the muon 
detector. Most of the high \pt muons go out of the CMS detector. Due to the presence of a magnetic field, 
trajectories of muon are bent in the CMS detector. The number of hits in the tracker, energy deposit
in the calorimeter and the hits in the muon chambers (DT, CSC, and RPC) are used in the Kalman filtering technique~\cite{FRUHWIRTH1987444}
to reconstruct muon candidates. There are several categories of muons~\cite{Chatrchyan:2012xi} described below:
\begin{itemize}
   \item {\bf{ Standalone Muons}}: These are reconstructed from the muon chambers only. Here, the
       Kalman filtering technique uses the hits from DT, CSC, and RPC to reconstruct standalone muons.
   \item {\bf{ Global Muons}}: The trajectory of a standalone muon is extrapolated up to the tracker. 
       The best pair consisting of the standalone muon and tracker track is selected using the
       Kalman filtering technique. Finally, these two tracks are combined to reconstruct the Global Muon.
       As the global muons are reconstructed starting from the outer part of the CMS detector
       (muon chambers) back to the inner part (strip tracker), this approach of reconstruction is called 
       {\em outside-in}.
   \item {\bf{ Tracker Muons}}: There is an {\em inside-out} approach to reconstruct muons, which 
       starts from the tracks reconstructed in the strip tracker to the energy deposits in the calorimeters
       up to the hits in the muon chamber. Muons reconstructed with this approach are called Tracker muons.
   \item {\bf{ RPC Muons}}: These muons are also reconstructed using the {\em inside-out} approach 
       where only RPC hits of the muon chamber are used~\cite{Goh:2014tra}. 
   \item {\bf{ Calo Muons}}: Some of the reconstructed muon tracks consist of energy deposits
       from the calorimeters. Such muons are called Calo muons.
\end{itemize}

\subsubsection{Identification and Selection}
In the event topology of our interest, there only one lepton (electron or muon) as shown in Figure~\ref{fig:feyn_diag_sig}. These are the 'prompt' lepton coming
directly from the electroweak decay of W bosons.However, there could 'extra'
leptons that come from other sources such as misidentification of light flavored
jets or charged pion decay in case of muons. Medium identification criteria as
recommended by the muon POG are applied to select prompt muons. These are listed
in Table~\ref{tab:muonSel}. The efficiency of medium muon ID as a function of
\pt and $\eta$ is shown in Figure~\ref{fig:muonIsoIDEff} for data and MC.
From this figure, it can be seen that the efficiency at lower \pt is relatively
low compared to that from high \pt. However there is a sudden drop in the
efficiency for data from era BCDEF in the \pt$ > 120$ GeV region. These 
efficiency plots are provided by the muon POG~\cite{muSF}. 
\begin{table}
  \caption{Selection and veto cuts applied on muon.}
 \begin{center}
 \begin{tabular}{ccc}\hline\hline
 Variable & Selection & Veto\\ \hline\hline
 \pt (GeV) & $> 26 $ & $> 15$\\
 $|\eta|$ & $< 2.4$ & $< 2.4$ \\
 Global or tracker muon & Yes & Global \\
 Normalised $\chi^2$ & $< 3$ & \\
 $\chi^2$ local position & $< 12$ & \\
 Tracker kink & $< 20$ &\\
 Segment compatibility & $ > 0.303 $ & \\
 $|d_\mathrm{xy}(\mathrm{vertex})|$ (cm) & $< 0.05$ & \\
 $|d_\mathrm{z}(\mathrm{vertex})|$ (cm) & $< 0.2$ & \\
 Inner track valid fraction & $> 0.8$ & \\
 $I_{\rm rel}^\mu~$ & $< 0.15$ & $< 0.25$\\\hline
 \end{tabular}
 \end{center}
 \label{tab:muonSel}
 \end{table}
\begin{figure}
    \centering
    \subfigure[\label{subfig:muonIDEffPt}]
    {\includegraphics[width=0.49\linewidth]{Image/EffAndSF/muonIDEffPt.pdf}}
    \subfigure[\label{subfig:muonIDEffEta}]
    {\includegraphics[width=0.49\linewidth]{Image/EffAndSF/muonIDEffEta.pdf}}
    \vfil
    \subfigure[\label{subfig:muonIsoEffPt}]
    {\includegraphics[width=0.49\linewidth]{Image/EffAndSF/muonIsoEffPt.pdf}}
    \subfigure[\label{subfig:muonIsoEffEta}]
    {\includegraphics[width=0.49\linewidth]{Image/EffAndSF/muonIsoEffEta.pdf}}
    \caption{ The distribution of efficiencies for medium muon ID and tight relative
    isolation (with medium ID) as a function of \pt and $\eta$ of muon from MC and data
    from different eras.}
    \label{fig:muonIsoIDEff}
\end{figure}

\subsubsection{Rochester correction}
Due to misalignment of the magnetic field and azimuth dependent in-efficiency of 
the muon detector, the efficiency of the muon needs to be corrected. The efficiency
depends on the charge, \pt, $\eta$ and $\phi$ of muon. These corrections referred as 
Rochester correction are applied on data as well as MC samples. The muon \pt is
corrected using Rochester correction tools at 13 TeV~\cite{Bodek:2012id}.
The Rochester scale factors depend on the \pt, $\eta$, $\phi$ and charge of the 
muon. Selection criteria (Table~\ref{tab:muonSel}) on muon are applied after 
Rochester correction.

\subsubsection{Isolation}
A muon track in the pp collision is not necessarily isolated i.e., it can be 
surrounded by other particles. On the other hand, muons from the W decay are 
expected to be isolated. To achieve this, we apply a criterion on the relative
isolation, defined as the \pt sum of all particles excluding the muon within a 
cone of radius 0.4 around the muon direction divided by the muon \pt i.e.
$I_{\rm rel}^\mu=\sum_{i\neq\mu}\pt^i/\pt^\mu$. We use a more specific definition given by  
\begin{equation}
\label{eq:mu_iso}
I_{\rm rel}^\mu=\frac{\sum\pt^{\rm ch}+{\rm max}[(\sum E_T^\gamma+\sum E_T^{\rm neut}-0.5\times\sum\pt^{\rm chPU}),0]}{\pt^\mu},
\end{equation}
where \pt$^{\rm ch}$ is the transverse momentum of charged hadrons; $E_T^\gamma$ and
$E_{T}^{\rm neut}$ are the transverse energy of photons and neutral hadrons; and
\pt$^{\rm chPU}$ is the transverse momentum of charged hadrons associated with the
pileup vertices. The factor $0.5$ takes into account the neutral tracks from the
leading vertex and charged tracks from the pileup vertices~\cite{Sirunyan:2018fpa}.
A tight muon isolation cut of $I_{\rm rel}^\mu < 0.15$ is applied. The efficiency
as a function of \pt and $\eta$ of tight muon isolation with medium ID is shown in
Figure~\ref{fig:muonIsoIDEff} \cite{muSF}.

%--------------------------------------
% Electron Reconstruction
%--------------------------------------
\subsection{Electron }
\subsubsection{Reconstruction}
Electrons are reconstructed using the PF algorithm~\cite{CMS-PAS-PFT-09-001, CMS-PAS-PFT-10-001} 
based on the tracks in the tracker and energy deposits in the electromagnetic calorimeter
~\cite{Khachatryan:2015hwa}. Being a charged particle electrons radiate as they travel inside the huge
magnetic field of the CMS. Thus the energy deposited by an electron in the ECAL is
spread in $\eta$ and $\phi$ direction across several crystals. The energy deposited
in various crystals are clustered using "hybrid" algorithm in the barrel and "multi-5x5"
in endcap part of the ECAL. For given hits in the innermost layer of the tracker, the 
Kalman Filter technique is used to find hits in the next layer of the tracker. The hits collected
from all layers of the tracker is mapped to the energy deposit in the ECAL to form 
an electron candidate. To take care of the effect of bremsstrahlung, the "Gaussian sum 
filter" method is used to fit the tracks. The electron charge is estimated from the 
curvature of the electron track and by matching associated KF track with GSF tracks. The vector joining (SC, beam-spot) and (SC, first hit of the GSF track) in $\phi$ is also used for
electron charge estimation. The measurements from tracker and ECAL are combined to estimate
the electron momentum~\cite{Khachatryan:2015hwa}.

\subsubsection{Identification and Selection}
To select prompt electrons, medium cut-based ID as listed in Table~\ref{tab:eleSel} is
used. Electrons with \pt$ > 30$\,GeV and $|\eta| <$ 2.5 are excluded. 
\begin{table}
  \caption{Selection and veto cuts applied on electron.}
 \label{tab:eleSel}
 \centering
\begin{adjustbox}{max width=\textwidth}
 \begin{tabular}{ccccc}\hline\hline
 Variable & \multicolumn{2}{c}{Selection} & \multicolumn{2}{c}{Veto}\\
  & $|\eta_{\rm sc}|\leq 1.479$ & $|\eta_{\rm sc}|>1.479$ & $|\eta_{\rm sc}|\leq 1.479$ & $|\eta_{\rm
 sc}|>1.479$\\\hline\hline
 \pt (GeV) & $> 30$ & $> 30$ & $> 15$ & $> 15$\\
 $|\eta|$ & $< 2.5$ & $< 2.5$ & $ -$ & $- $\\
 full $5\times 5$ $\sigma_{i\eta i\eta}$    & $< 0.00998$   & $< 0.0298$    & $< 0.0115 $   & $<0.037$\\
 $|\Delta\eta({\rm sc},{\rm track})|$       & $< 0.00311$   & $< 0.00609$   & $< 0.00749$   & $<0.00895$\\
 $|\Delta\phi({\rm sc},{\rm track})|$       & $< 0.103$     & $< 0.045$     & $< 0.228$     & $< 0.213$\\
 $E_{{\rm had}}/E_{{\rm em}}$               & $< 0.253$     & $< 0.0878$    & $< 0.356$     & $<0.211$\\
 $I_{\rm rel}^e$                            & $< 0.0821$    & $< 0.0695$    & $< 0.175$    & $<0.159$\\
 $|\frac{1}{E}-\frac{1}{p}|$ (GeV)$^{-1}$   & $< 0.134$     & $< 0.13$      & $< 0.299$ &$<0.15$\\
 Missing inner hits                         & $\le 1$       & $\le 1$       & $\le 2$       & $\le 3$\\
 Conversion veto                            & Yes           & Yes           & Yes           & Yes\\
 $|d_{xy}({\rm vertex})|$ (cm)              & $< 0.05$      & $< 0.10$       &$<0.05$      & $<0.05$\\
 $|d_z({\rm vertex})|$ (cm)                 & $< 0.10$      & $< 0.20$       & $<0.1$      &$<0.1$\\\hline
 \end{tabular}
\end{adjustbox}
 \end{table}

A brief description of the variables listed in Table~\ref{tab:eleSel} is given below:
\begin{itemize}
    \item full $5\times 5$ $\sigma_{i\eta i\eta}$ is defined as~\cite{Khachatryan:2015hwa} 
    \begin{equation}
    \sigma_{i\eta i\eta}=\sqrt{\sum(0.0175\times\eta_i+\eta^{\rm seed}_{\rm cryst}-\bar{\eta}_{5\times 5})^2\times w_i]/\sum w_i}, \quad w_i = 4.2 + \log(E_i/E_{5\times 5})
    \label{eq:sIeIe}
    \end{equation}
    The summation is carried over the $5\times 5$ matrix around the highest $E_T$ crystal of the supercluster.
    \item $|\Delta\eta({\rm sc},{\rm track})|$: Difference in $\eta$ of the supercluster and the electron track.
    \item $|\Delta\phi({\rm sc},{\rm track})|$: Difference in $\phi$ the supercluster and the electron track.
    \item $E_{\rm had}/E_{\rm em}$: Ratio of the electron energy deposited in the HCAL and ECAL.
    \item $|\frac{1}{E}-\frac{1}{p}|$: $E$ is the energy of the supercluster and $p$ is the track momentum at
    the point of closest approach from the PV.
    \item Missing inner hits: Number of missing hits in the strip tracker.
    \item $|d_{xy}({\rm vertex})|$: Distance of closest approach of the electron track from the PV in the $xy$ plane.
    \item $|d_{z}({\rm vertex})|$: Difference in the $z$-coordinate of the PV and the electron track.
\end{itemize}
The relative isolation $I_{\rm rel}^e$ and conversion veto of Table~\ref{tab:eleSel}
are discussed in Sections~\ref{s:ele_iso} and \ref{s:ele_conv_rej}. The efficiency of medium and 
veto electron ID as a function of \pt and $\eta$ is shown in Figure~\ref{fig:eleIDEff} \cite{eleSF}.
\begin{center}
\begin{figure}
   \subfigure[Efficiency for medium ID]
   {\includegraphics[width=0.49\linewidth]{Image/EffAndSF/eleMediumIDEff_mc2D.pdf}}
   \subfigure[Efficiency for veto ID]
   {\includegraphics[width=0.49\linewidth]{Image/EffAndSF/eleMediumIDEff_data2D.pdf}}
\caption{ The distribution of efficiency for medium and veto electron ID.} 
\label{fig:eleIDEff}
\end{figure}
\end{center}

\subsubsection{Isolation}
\label{s:ele_iso}
The relative isolation of an electron is given by:
\begin{equation}
\label{eq:ele_iso}
I_{\rm rel}^e= \frac{\sum\pt^{ch}+{\rm max}[(\sum E_T^\gamma+\sum E_T^{\rm neut}-\rho\times A_{\rm eff}),0]}{\pt^e},
\end{equation}
where $\rho$ is as defined in Eq.~(\ref{eq:rho}) and $A_{\rm eff}$ is the effective
area. (To get $\rho$ for $I_{\rm rel}^e$ from {\em MINIAOD}, {\em fixedGridRhoFastjetAll} tag is used.)
These two quantities are used to correct electron isolation from pileup.
The effective area for different $\eta_{\rm sc}$ are shown in Table~\ref{tab:EA}. The
tight isolation cut of $I_{\rm rel}^e < 0.0821 (0.0695)$ is applied on electrons from
the barrel (endcap) regions.
\begin{table}
 \caption{Effective area for different $\eta_{\rm sc}$ range.}
 \begin{center}
 \begin{tabular}{cc}\hline\hline
     Super-cluster pseudorapidity $(\eta_{\rm sc}$) & $A_{\rm eff}$ \\ \hline\hline
     $0.0 \leq \eta_{\rm sc} < 1.0 $ & $0.1703$ \\
     $1.0 \leq \eta_{\rm sc} < 1.5 $ & $0.1715$ \\
     $1.5 \leq \eta_{\rm sc} < 2.0 $ & $0.1213$ \\
     $2.0 \leq \eta_{\rm sc} < 2.2 $ & $0.1230$ \\
     $2.2 \leq \eta_{\rm sc} < 2.3 $ & $0.1635$ \\
     $2.3 \leq \eta_{\rm sc} < 2.4 $ & $0.1703$ \\
     $2.4 \leq \eta_{\rm sc} $ & $0.2393$ \\\hline
 \end{tabular}
 \end{center}
 \label{tab:EA}
 \end{table}

\subsubsection{Conversion rejection}
\label{s:ele_conv_rej}
Some of the reconstructed electrons are end products of photon conversion inside the tracker.
While selecting prompt electrons these secondary electrons need to be rejected.
The tag used for conversion rejection is {\em reducedEgamma:reducedConversions}.
As the secondary electrons don't have hits in the innermost layer of the tracker~\cite{Khachatryan:2015hwa},
one can use this feature to reject them.
By fitting the electron tracks, these secondary electrons are also rejected ~\cite{Khachatryan:2015hwa}.

%--------------------------------------
% Jet Reconstruction
%--------------------------------------
\subsection{Jet}
\label{s:jet_reco}
\subsubsection{Reconstruction}
Due to color confinement~\cite{Polyakov:1976fu}, the quarks and gluons produced in pp collisions cannot exist in the free states.
Instead, they hadronise into a cluster of colorless particles such as hadrons, leptons, and photons.
We form a jet out of these particles, emanating in a cone around the initial direction quarks and gluons.
In CMS the PF algorithm is used to reconstructs all possible particles that can be inside a jet.
Subsequently, the clustering of, or the reconstruction of jets from, these particles are done with the
anti-$k_T$ algorithm~\cite{Cacciari:2008gp}. 

\subsubsection{Identification and Selection}
\label{ss:jet_id}
The recommended \cite{jetID} selection criteria applied on jets are listed in
Table~\ref{tab:jetSel}. The identification of b and c-jet are discussed in Sections~\ref{s:bTag} and \ref{s:cTag}. 
\begin{table}
  \caption{Selection criteria applied on jets.}
 \begin{center}
 \begin{tabular}{cc}\hline\hline
 Variable & selection \\ \hline\hline
 \pt (GeV) & $> 25$ \\
 $|\eta|$ & $< 2.4$  \\
 Neutral hadron energy fraction & $<0.99$ \\
 Neutral electromagnetic energy fraction & $<0.99$\\
 Number of constituent & $ > 1$\\ 
 Charged hadron energy fraction & $ >0$ \\
 Charged hadron multiplicity & $> 0$ \\
 Charged hadron electromagnetic energy fraction & $ < 0.99$\\\hline
 \end{tabular}
 \end{center}
 \label{tab:jetSel}
 \end{table}
The charged hadron subtraction technique ~\cite{CMS-PAS-JME-14-001} is used to 
take care of the pileup effects. It ignores the charged particles coming from 
the pileup vertices. The jet energy in data and simulated samples are corrected 
to account for the non-linear response of the ECal and HCal. A set of corrections
are applied in a factored manner, that is the output of previous serves as the 
input to the next step. The raw \pt of a jet is corrected at every step and the
corrected \pt is used in the subsequent analysis. These set of corrections are 
known as \verb|L1FastJet| (pileup correction through offset energy density)
\verb|L2RelativeL3Absolute| (detector response corrections), and 
\verb|L2L3Residual| (residual corrections on data). The first two are applied 
on simulated samples whereas all the three are applied on data. All these
corrections are applied while processing the MiniAOD datasets listed in the 
Table (\ref{tab:mcSample}, \ref{tab:dataSample}). Furthermore, the jet \pt in 
the simulation are smeared to have same resolution as that of data using the 
jet energy resolution (JER) scale factors. Before applying the selection cuts 
listed in Table~\ref{tab:jetSel}, the jet \pt is corrected as discussed in
Section~\ref{s:JEC}. 

\subsubsection{Cleaning}
The prompt leptons (electrons and muons) are required to be outside the jet cone.
This procedure is called jet cleaning, where we require the angular separation between the
jet and lepton, $\Delta R > 0.5$.

%--------------------------------------
% \MET Reconstruction
%--------------------------------------
\subsection{Missing Transverse Energy}
The missing transverse energy (\MET) is the negative vector sum of momenta of all objects
reconstructed with the PF algorithm.
Being a weakly interacting particle, neutrinos are not directly detected through the CMS
detector and thus contribute to \MET.
Note that this is not true only for the SM; there are some beyond-the-SM particles e.g.,
neutralinos in supersymmetry that could also contribute to \MET.
Nevertheless, as there is a neutrino in our signal events (see Figure~\ref{fig:feyn_diag_sig})
we expect large \MET due to its escape from the detector.
The \MET in MC and data can have different resolutions.
Therefore, \MET from MC events are smeared using the {\em Type-1} correction in which jet energy
correction is propagated to the \MET.
The events having \MET $>$ 20 GeV are selected.
