\subsection*{Luminosity}
 The luminosity is calculated using the following command
\begin{align}
\text{brilcalc  lumi  -b  "STABLE BEAMS" --normtag DATACERT.json} \nonumber \\  
\text{-u  /fb -i processedLumis.json}
\label{cmd:lumiCalc}
\end{align}

\subsection*{Pileup}
The pileup distribution in the data is calculated using the following command 
\begin{align}
\text{pileupCalc.py -i JSONfile \texttt{--}inputLumiJSON pileup\_latest.txt  
\texttt{--}calcMode true} \nonumber \\
\text{\texttt{--}minBiasXsec 69200 \texttt{--}maxPileupBin 101 \texttt{--}numPileupBins 101  puHist.root}
\label{cmd:pileupCalc}
\end{align}
where \verb|JSONfile| is the golden JSON file which contains good lumi section.

\subsection*{Fit Diagnostics}
The following commands are used for maximum likelihood fit inside the \textit{combine} tool
\begin{subequations}
\begin{align}
\text{combine t2wDataCard.root -m 100 -M FitDiagnostics \texttt{--}expectSignal 1 \nonumber} \\
	\text{\texttt{--}redefineSignalPOIs BR \texttt{--}setParameterRanges BR=0,0.10 \texttt{--}plots} \nonumber \\
	\text{\texttt{--}saveShapes \texttt{--}saveWithUncertainties  \texttt{--}saveNormalizations} \nonumber \\
	\text{\texttt{--}cminDefaultMinimizerStrategy 0}
\end{align}    
\begin{align}
\text{python diffNuisances.py -a fitDiagnostics.root \texttt{--}poi=BR -g fitDiag.pdf}
\end{align}
\label{cmd:fitDiag}
\end{subequations}

\subsection*{Goodness of Fit}
The following commands are used to compute the GOF
\begin{subequations}
\begin{align}
\text{combine t2wDataCard.root -M GoodnessOfFit \texttt{--}algo saturated}
\end{align}    
\begin{align}
\text{combine t2wDataCard.root -M GoodnessOfFit \texttt{--}algo saturated -t 1000 -s -1}
\end{align}
\label{cmd:GOF}
\end{subequations}

\subsection*{Impact of Parameters}
Following commands are used to compute the impact of NPs
\begin{subequations}
\begin{align}
\text{combineTool.py -M Impacts -d t2wDataCard.root -m 100 \texttt{--}doInitialFit} \nonumber \\
\text{\texttt{--}robustFit 1 \texttt{--}redefineSignalPOIs BR \texttt{--}setParameterRanges BR=0,0.1}
\nonumber \\
\text{ \texttt{--}cminDefaultMinimizerStrategy 0}
\end{align}    
\begin{align}
\text{combineTool.py -M Impacts -d t2wDataCard.root -m 100 \texttt{--}doFit} \nonumber \\
\text{\texttt{--}robustFit 1 \texttt{--}redefineSignalPOIs BR \texttt{--}setParameterRanges BR=0,0.1}
\nonumber \\ 
\text{\texttt{--}cminDefaultMinimizerStrategy 0 \texttt{--}parallel 50}
\end{align}
\begin{align}
\text{combineTool.py -M Impacts -d t2wDataCard.root -m 100 -o nuisImpactJSON}
\end{align}
\begin{align}
\text{plotImpacts.py \texttt{--}cms-label " Internal" -i nuisImpactJSON -o nuisImpactPDF}
\end{align}
\label{cmd:impact}
\end{subequations}

\subsection*{Limit}
Following commands are used to compute the limit
\begin{subequations}
\begin{align}
\text{text2workspace.py  DATACARD.txt -o t2wDATACARD.root}
\nonumber \\ 
\text{-P HiggsAnalysis.CombinedLimit.ChargedHiggs:brChargedHiggs}
\end{align}    

\begin{align}
\text{combine \texttt{--}rAbsAcc 0.000001 t2wDATACARD.root -M AsymptoticLimits}
\end{align}
\label{cmd:limit}
\end{subequations}
where \verb|DATACARD.txt| is the input data card file which contains the name 
of the histogram for each process, and information about the statistical and
systematics uncertainties.

\subsection*{Scan Parameters}
The $2\Delta NLL$ as a function of NPs is computed using the following command
\begin{align}
\text{combine -M MultiDimFit -d t2wDataCard.root -m 100 \texttt{--}redefineSignalPOIs BR} \nonumber \\
\text{\texttt{--}setParameterRanges BR=0.0,0.10 -P nameNP \texttt{--}trackParameters BR} \nonumber \\ 
\text{\texttt{--}algo=grid points 20 -n nameNP}
\label{cmd:deltaNLL}
\end{align}
