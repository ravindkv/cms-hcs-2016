
%%%%%%%%%%%%%%%%%%%%%%%%%%%%%%%%%%%%%%%%%%%%%%%%%%%%%%%%
\subsection{Data Samples}
The reconstructed data is broadly divided into different event topologies. Only those 
datasets matching with event topology under consideration are used for the analysis. 
Such approach help in optimising the computing resources.
The single muon and single electron data sample used for this analysis are shown in
Table (\ref{tab:dataSample}). The data was collected in 2016 corresponding to Run number 
273158 to 284044. To select good quality data, where most of the sub-detectors of the CMS 
operates in a normal condition; the following lumi mask, as recommended by the PdMV group is applied:
\verb|Cert_271036-284044_13TeV_23Sep2016ReReco_Collisions16_JSON.txt|. The integrated luminosity
after applying the mask reduces marginally from 37.0\fbinv to
35.9\fbinv. The luminosity is calculated using the Command (\ref{cmd:lumiCalc}). 
The global tag \verb|80X_dataRun2_2016SeptRepro_v7| is used to process the {\em MINIAOD}
data in the \verb|CMSSW_8_0_28| for the production of the {\em ntuples}. The same version
of the \verb|CMSSW| is used to produce {\em ntuples} for data, background, and signal samples.

\begin{table}
\centering
    \caption{Muon and Electron data samples recorded in 2016 with an integrated luminosity 35.9
    \fbinv. The re-reco campaign was used for Run B to Run G. For Run H, prompt reconstructions were used. 
    The luminosity is calculated using {\bf{brilcalc}} tool. Where M = \/MINIAOD.} 
\label{tab:dataSample}
\begin{adjustbox}{max width=\textwidth}
\begin{tabular}{cccc} \hline\hline                                                                                                                  
    {\bf{Dataset}} & {\bf{$\boldmath{L}_{int}$ (\fbinv)}} & {\bf{Run}} & {\bf{Events}}
    \\\hline\hline
 \verb|/SingleMuon/Run2016B-03Feb2017_ver2-v2/M|    & 5.78 & 273158-275376  & 154054252  \\[0.1cm]
 \verb|/SingleMuon/Run2016C-03Feb2017-v1/M|         & 2.57 & 275657-276283  & 64718679   \\[0.1cm]
 \verb|/SingleMuon/Run2016D-03Feb2017-v1/M|         & 4.25 & 276315-276811  & 96657799  \\[0.1cm]
 \verb|/SingleMuon/Run2016E-03Feb2017-v1/M|         & 4.01 & 276831-277420  & 87362752   \\[0.1cm]
 \verb|/SingleMuon/Run2016F-03Feb2017-v1/M|         & 3.10 & 277981-278808  & 65047318   \\[0.1cm]
 \verb|/SingleMuon/Run2016G-03Feb2017-v1/M|         & 7.54 & 278820-280385  & 147945745  \\[0.1cm]
 \verb|/SingleMuon/Run2016H-03Feb2017_ver2-v1/M|    & 8.40 & 281613-284035  & 166591136  \\[0.1cm]
 \verb|/SingleMuon/Run2016H-03Feb2017_ver3-v1/M|    & 0.21 & 284036-284044  & 4389914    \\\hline
     {\bf{Total}}                                    & {\bf{35.86}} & {\bf{273158-284044}}  &
     {\bf{786767595}}    \\[0.1cm]\hline                                                   

 \verb|/SingleElectron/Run2016B-03Feb2017_ver2-v2/M|    & 5.78 & 273158-275376  & 237366108  \\[0.1cm]
 \verb|/SingleElectron/Run2016C-03Feb2017-v1/M|         & 2.57 & 275657-276283  & 91591087   \\[0.1cm]
 \verb|/SingleElectron/Run2016D-03Feb2017-v1/M|         & 4.25 & 276315-276811  & 146495223  \\[0.1cm]
 \verb|/SingleElectron/Run2016E-03Feb2017-v1/M|         & 4.01 & 276831-277420  & 113169852  \\[0.1cm]
 \verb|/SingleElectron/Run2016F-03Feb2017-v1/M|         & 3.10 & 277981-278808  & 70143321   \\[0.1cm]
 \verb|/SingleElectron/Run2016G-03Feb2017-v1/M|         & 7.54 & 278820-280385  & 152098617  \\[0.1cm]
 \verb|/SingleElectron/Run2016H-03Feb2017_ver2-v1/M|    & 8.40 & 281613-284035  & 123900510  \\[0.1cm]
 \verb|/SingleElectron/Run2016H-03Feb2017_ver3-v1/M|    & 0.21 & 284036-284044  & 3189661    \\\hline
   {\bf{Total}}                                          & {\bf{35.86}} & {\bf{273158-284044}}  & {\bf{937954379}}    \\\hline
 \end{tabular}          
\end{adjustbox}









\end{table}

\subsection{Simulated Samples}
The centrally produced official simulated samples are used in this analysis. 
Production of most of the samples (except 
\verb|QCD_Pt-15to20_EMEnriched|) started with campaign 
\verb|RunIISummer16MiniAODv2| and final reconstruction was done with 
campaign \verb|PUMoriond17| using Global Tag \\ 
\verb|80X_mcRun2_asymptotic_2016_TrancheIV_v6|. 
The simulated samples used in this analysis are shown in Table (\ref{tab:mcSample}) 
with corresponding cross-section and number of simulated events. 
The k-factor is the ratio of cross-sections from next-to-leading-order 
and leading-order, k$_\text{f} =\sigma_{\text{NLO}}/\sigma_{\text{LO}}$.

The signal and background simulated samples are generated using \MGvATNLO~\cite{Alwall:2011uj, Alwall:2014hca}
as well as \POWHEG~\cite{Frixione:2007vw, Nason:2004rx, Alioli:2010xd} generators at parton level. 
These parton level events are hadronised using \PYTHIA~\cite{Sjostrand:2006za, Sjostrand:2007gs}. 
The hadronised events were tuned using  CUETP8M1~\cite{CMS-PAS-TOP-16-021} which are then passed
through \GEANTfour~\cite{Agostinelli:2002hh} for detector simulation. Finally, the events are
reconstructed after complete detector simulation. In summary, the 
following steps are followed to produce simulated samples for a single process:
\begin{enumerate}
    \item {\bf{GEN-SIM}}: 
        \begin{itemize}
            \item GEN: Physics process generation. 
            \item SIM: Detector simulation.
        \end{itemize}
    \item {\bf{RECO}}: 
        \begin{itemize}
            \item DIGIPREMIX\_S2: \\
                \verb|https://twiki.cern.ch/twiki/bin/view/CMSPublic/SWGuideSimulation|, 
            \item DATAMIX: \\\verb|https://twiki.cern.ch/twiki/bin/view/CMSPublic/SWGuideSimulation|,
            \item L1: L1 trigger emulation,
            \item DIGI2RAW: Conversion of the digitised data format to raw data format, 
            \item HLT: Applying high level trigger,
            \item RAW2DIGI: Conversion of the raw data format to DIGI data format,
            \item RECO: Reconstruction of physics object,
            \item EI: High-level reconstruction step used in validation.
        \end{itemize}
    \item {\bf{MINIAOD}}: Re-reconstruction of physics object.
\end{enumerate}

The \ttjets channel is the most significant irreducible background for this analysis. It
contributes around 94\% of the total backgrounds in the signal region. The parton level events 
for \ttjets samples of Table~\ref{tab:mcSample} are generated using 
\POWHEG~\cite{Frixione:2007vw, Nason:2004rx, Alioli:2010xd} at next-to-leading (NLO) order. The 
\verb|NNPDF30_nlo_as_0118| ~\cite{Ball:2014uwa} parton distribution function (PDF) was used 
for this purpose. The partonic events were then hadronised using 
\PYTHIA~\cite{Sjostrand:2006za, Sjostrand:2007gs}. To simulate this channel at higher orders and 
to take care of non-perturbative effects, \ttjets samples were tuned with 
CUETP8M1~\cite{CMS-PAS-TOP-16-021}. The NNLO cross section for \ttjets process is estimated to be 
$831.76 \pm^{20}_{29}(\text{scale}) \pm 35 (\text{PDF} + \alpha_s)$ pb \cite{Beneke:2011mq}. 

Initially, the \ttjets sample was simulated using MadGraph generator (\\ 
\verb|/TTJets_TuneCUETP8M1_13TeV-madgraphMLM-pythia8/year+M|, where \verb|year| and \verb|M| 
are defined in Table (\ref{tab:mcSample}). However, MadGraph-MLM sample was found to be 
improperly modeled for high jet multiplicity due to which there was a surplus of 5\% (4\%) 
simulated events w.r.t. observed data for muon (electron) + jets channel in the signal region as 
shown in Figure~\ref{fig:MGvsPG}. In view of this, the POWHEG \ttjets sample is used in this analysis.
Comparison of POWHEG and Madgraph \ttjets sample is described in Appendix~\ref{a:appendMGvsPG}.

Single \PQt quark samples are shown in Table~\ref{tab:mcSample}, where a \PQt quark is produced with 
jets in \PQt-channel (\verb|ST_t|), \PQs-channel (\verb|ST_s|), and \PQt\PW-channel (\verb|ST_tW|). The
\verb|ST_tW| and \verb|ST_t| samples are produced using \POWHEG + \PYTHIA and CUETP8M1, where \verb|ST_tW| has 
5-flavor scheme whereas \verb|ST_t|, has 4-flavor scheme and further, the \PQt quark decays 
inclusively. The \verb|ST_s| are generated using \MGvATNLO~\cite{Alwall:2014hca} in the 4-flavor 
scheme and was hadronised using \PYTHIA. The NLO cross section for single \PQt process 
\cite{Aliev:2010zk, Kant:2014oha} are shown in Table~\ref{tab:mcSample}.

The inclusive \wjets and \dyjets samples were generated using \MGvATNLO and are hadronised using 
\PYTHIA. The MLM~\cite{ Alwall:2007fs} technique was used to take care of double counting. There
are exclusive \PW + n jets and $\PZ/\PGg$ + n jets samples available for n = 1, 2, 3, 4. 
The inclusive (\wjets and \dyjets) and exclusive ($\PW/\PZ/\PGg$ + n jets) samples have our event 
topology \ie 4 jets + 1 lepton + \MET. Therefore, the inclusive and exclusive samples are added 
appropriately to have more statistics. To avoid double counting, an 
event weight based on luminosity and number of particles (NUP) 
including incoming, intermediate, and outgoing at generator level is 
applied to each sample. After that, the final event yields from each of
them are added linearly. The event weight for NUP = 5, 6, 7, 8, and $\geq$9 is
 $L_{data}/L_{\wjets}$, $L_{data}/(L_{\wjets} + L_{\PW + \text{1 jet}})$, 
 $L_{data}/(L_{\wjets} + L_{\PW + \text{2 jet}})$, 
 $L_{data}/(L_{\wjets} + L_{\PW + \text{3 jet}})$,
 $L_{data}/(L_{\wjets} + L_{\PW + \text{4 jet}})$, respectively.
 Where, for example, $L_{\wjets}$ is the generated luminosity (number 
 of events divided by the cross section) of \wjets sample. A similar 
 procedure is followed to add $\PZ/\PGg$ + n jets with \dyjets. However,
 the NUP variable does not count the intermediate photon as the status
 of photon is not displayed in the LHE event file. Hence it is
 incremented by 1 when a \PZ boson is not the intermediate particle.
 The NLO cross section for these processes are shown in Table~\ref{tab:mcSample}.

The vector boson fusion process (WW, WZ, and ZZ) samples are generated and hadronised 
using \PYTHIA and are tuned using CUETP8M1. The NLO cross section for these samples are shown 
in Table~\ref{tab:mcSample}.

The muon (electron) enriched QCD multijet samples are used for \mujets (\ejets) channel. These 
samples are generated using \PYTHIA and CUETP8M1. After multiplying with filter efficiency, the 
NLO cross sections are shown in Table~\ref{tab:mcSample}. From this table, it can be seen that the 
luminosity for these samples are significantly 
smaller as compared to the observed luminosity and therefore, a data-driven approach is used to 
make more precise estimation of QCD multijet background.

The charged Higgs signal samples were generated using \MGvATNLO, and hadronized using \PYTHIA. The 
signal sample for several mass points in the range of 80 to 160 \GeV (80, 90, 100, 120, 140, 150, 
155, 160) are generated for the search for charged Higgs. The cross section 
for the signal is 831.76$\times$ 0.2132 pb, where 831.76 \unit{pb} is the inclusive 
(fully-hadronic, fully-leptonic, semi-leptonic) \ttjets production cross section and factor 
0.2132 is the branching fraction of $W^-\to l^- \bar{\nu_l}$ (where $l = \mu, e$, $\tau$ is not 
considered in this analysis) \cite{Beringer:1900zz}. The factor 0.2132 is multiplied because \ttbar decays 
semi-leptonically ($(t\to H^+ b, H^+\to c\bar{s}), (\bar{t}\to W^-\bar{b}, W^- \to l^-\bar{\nu_l})$), 
where $l = \mu, e$ in the charged Higgs signal samples. Furthermore, the signal events are scaled by 
the maximum observed upper limit obtained at 8 \TeV. The upper observed limit on \brThb is 6.5\% for
the $m_{H+}$ = 90 \GeV\cite{Khachatryan:2015uua}. Therefore, the signal samples are scaled by a factor 
$2\times 0.065\times (1-0.065) = 0.12155$ in every plot and table except in the data cards used for 
limit computation. 
\begin{table}
\caption{Signal and background simulated samples. Where \\  
\small{
run = { \em RunIISummer16MiniAODv2-PUMoriond17\_80X\_mcRun2\_asymptotic\_2016\_TrancheIV\_v6}, \\
run2= { \em RunIISpring16MiniAODv2-FlatPU8to37HcalNZSRAW\_withHLT\_80X\_mcRun2\_asymptotic\_v14-v1},\\
M = /MINIAODSIM, yext1\_v2 = { \em run\_ext1-v2, yext1\_v1 = run\_ext1-v1},year = run-v1.
}
}
\label{tab:mcSample}
\centering

\begin{adjustbox}{max width=\textwidth}
\begin{tabular}{cccc} \hline\hline
    {\bf{MC Dataset}} & {\bf{$\sigma$}} (pb)(Order) &{\bf{$k_f$}}& {\bf{Events}}\\\hline\hline
    \verb|/TT_TuneCUETP8M2T4_13TeV-powheg-pythia8/+year+M|     
    &$831.76^{+20}_{-29} \pm 35$(NNLO) & --- & 77081156 \\ [0.1cm] 
    \verb|/ST_tW_antitop_5f_inclusiveDecays_13TeV-powheg| \\ 
    \verb|-pythia8_TuneCUETP8M1/+yext1_v1+M|  
    &$71.7 \pm 1.80 \pm 3.40$(NLO) & --- &  6933094   \\[0.1cm]
    \verb|/ST_t-channel_antitop_4f_inclusiveDecays_13TeV-powhegV2|\\
    \verb|-madspin-pythia8_TuneCUETP8M1/+year+M|  
    &$80.95^{+5.8}_{-5.2} \pm 0.16$(NLO) & ---  & 38811017  \\ [0.1cm]
    \verb|/ST_s-channel_4f_InclusiveDecays_13TeV-amcatnlo-pythia8/+year+M|  
    &$10.32^{+0.6}_{0.5} \pm 0.01$(NLO) & ---  & 2989199  \\[0.1cm] \hline
    
    \verb|/WJetsToLNu_TuneCUETP8M1_13TeV-madgraphMLM-pythia8/+year+M|       
    &$50690 \pm 389.1$(LO) & 1.21  & 29181900  \\[0.1cm]
    \verb|/W1JetsToLNu_TuneCUETP8M1_13TeV-madgraphMLM-pythia8/+year+M|   
    &$9493 \pm 25.52$(LO)      & 1.21  & 44813600  \\[0.1cm]
    \verb|/W2JetsToLNu_TuneCUETP8M1_13TeV-madgraphMLM-pythia8/+year+M|      
    &$3120 \pm 78.5$(LO)   & 1.21  & 29878415  \\[0.1cm]
    \verb|/W3JetsToLNu_TuneCUETP8M1_13TeV-madgraphMLM-pythia8/+year+M|      
    &$942.3 \pm 36.8$(LO)  & 1.21  & 19798117  \\[0.1cm]
    \verb|/W4JetsToLNu_TuneCUETP8M1_13TeV-madgraphMLM-pythia8/+year+M|      
    &$524.2 \pm 23.6$(LO)  & 1.21  & 9170576  \\[0.1cm]\hline

    \verb|/DYJetsToLL_M-50_TuneCUETP8M1_13TeV-madgraphMLM-pythia8/+yext1_v2+M| 
    &$4895 \pm 41$(LO)    & 1.17  & 48103700  \\[0.1cm]
    \verb|/DY1JetsToLL_M-50_TuneCUETP8M1_13TeV-madgraphMLM-pythia8/+year+M|
    &$1016 \pm 16.8$(LO)  & 1.17  & 62079400  \\[0.1cm]
    \verb|/DY2JetsToLL_M-50_TuneCUETP8M1_13TeV-madgraphMLM-pythia8/+year+M|     
    &$331.3\pm 8.5$(LO)   & 1.17  & 19970551  \\[0.1cm]
    \verb|/DY3JetsToLL_M-50_TuneCUETP8M1_13TeV-madgraphMLM-pythia8/+year+M|     
    &$96.6 \pm 3.9$(LO)   & 1.17  & 5856110  \\[0.1cm]
    \verb|/DY4JetsToLL_M-50_TuneCUETP8M1_13TeV-madgraphMLM-pythia8/+year+M|     
    &$51.4 \pm 2.5$(LO)   & 1.17  & 4197868  \\[0.1cm] \hline

    \verb|/WW_TuneCUETP8M1_13TeV-pythia8/+year+M|  
    &$118.7^{+2.5}_{-2.2}$(NNLO) & ---   & 994012  \\[0.1cm] 
    \verb|/WZ_TuneCUETP8M1_13TeV-pythia8/+year+M|  
    &$46.74^{+1.9}_{1.5}$(NLO)  & ---  & 1000000 \\[0.1cm] 
    \verb|/ZZ_TuneCUETP8M1_13TeV-pythia8/+year+M|  
    &$17.72^{0.6}_{0.4}$(NLO)  & ---  & 990064  \\[0.1cm] \hline
 \verb|/QCD_Pt-15to20_MuEnrichedPt5_TuneCUETP8M1_13TeV_pythia8/+year+M|    & 3819570(NLO) & ---  & 4141251   \\[0.1cm]
 \verb|/QCD_Pt-20to30_MuEnrichedPt5_TuneCUETP8M1_13TeV_pythia8/+year+M|    & 2960198(NLO) & ---  & 31475157  \\[0.1cm]
 \verb|/QCD_Pt-30to50_MuEnrichedPt5_TuneCUETP8M1_13TeV_pythia8/+year+M|    & 1652471(NLO) & ---  & 29954815  \\[0.1cm]
 \verb|/QCD_Pt-50to80_MuEnrichedPt5_TuneCUETP8M1_13TeV_pythia8/+year+M|    & 437504 (NLO) & ---  & 19806915  \\[0.1cm]
 \verb|/QCD_Pt-80to120_MuEnrichedPt5_TuneCUETP8M1_13TeV_pythia8/+year+M|   & 106033 (NLO) & ---  & 13786971  \\[0.1cm]
 \verb|/QCD_Pt-120to170_MuEnrichedPt5_TuneCUETP8M1_13TeV_pythia8/+year+M|  & 25190  (NLO) & ---  & 8042721   \\[0.1cm]
 \verb|/QCD_Pt-170to300_MuEnrichedPt5_TuneCUETP8M1_13TeV_pythia8/+year+M|  & 8654   (NLO) & ---  & 7947159   \\[0.1cm]
 \verb|/QCD_Pt-300to470_MuEnrichedPt5_TuneCUETP8M1_13TeV_pythia8/+year+M|  & 797    (NLO) & ---  & 7937590   \\[0.1cm]\hline

 \verb|/QCD_Pt-20to30_EMEnriched_TuneCUETP8M1_13TeV_pythia8/+year+M|      & 5352960(NLO) & ---  & 9218954    \\[0.1cm]
 \verb|/QCD_Pt-30to50_EMEnriched_TuneCUETP8M1_13TeV_pythia8/+year+M|      & 9928000(NLO) & ---  & 4730195    \\[0.1cm]
 \verb|/QCD_Pt-50to80_EMEnriched_TuneCUETP8M1_13TeV_pythia8/+year+M|      & 2890800(NLO) & ---  & 22337070   \\[0.1cm]
 \verb|/QCD_Pt-80to120_EMEnriched_TuneCUETP8M1_13TeV_pythia8/+year+M|     & 350000 (NLO) & ---  & 35841783   \\[0.1cm]
 \verb|/QCD_Pt-120to170_EMEnriched_TuneCUETP8M1_13TeV_pythia8/+year+M|    & 62964  (NLO) & ---  & 35817281   \\[0.1cm]
 \verb|/QCD_Pt-170to300_EMEnriched_TuneCUETP8M1_13TeV_pythia8/+year+M|    & 18810  (NLO) & ---  & 11540163   \\[0.1cm]
 \verb|/QCD_Pt-300toInf_EMEnriched_TuneCUETP8M1_13TeV_pythia8/+year+M|    & 1350   (NLO) & ---  & 7373633    \\[0.1cm]\hline

 \verb|/ChargedHiggsToCS_M080_13TeV-madgraph/+year+M|  & $0.2132 \times 831.76$(NNLO) & --- &996170   \\[0.1cm]
 \verb|/ChargedHiggsToCS_M090_13TeV-madgraph/+year+M|  & $0.2132 \times 831.76$(NNLO) & --- &994498   \\[0.1cm]
 \verb|/ChargedHiggsToCS_M100_13TeV-madgraph/+year+M|  & $0.2132 \times 831.76$(NNLO) & --- &987730   \\[0.1cm]
 \verb|/ChargedHiggsToCS_M120_13TeV-madgraph/+year+M|  & $0.2132 \times 831.76$(NNLO) & --- &990645   \\[0.1cm]
 \verb|/ChargedHiggsToCS_M140_13TeV-madgraph/+year+M|  & $0.2132 \times 831.76$(NNLO) & --- &952984   \\[0.1cm]
 \verb|/ChargedHiggsToCS_M150_13TeV-madgraph/+year+M|  & $0.2132 \times 831.76$(NNLO) & --- &992264   \\[0.1cm]
 \verb|/ChargedHiggsToCS_M155_13TeV-madgraph/+year+M|  & $0.2132 \times 831.76$(NNLO) & --- &976710   \\[0.1cm]
 \verb|/ChargedHiggsToCS_M160_13TeV-madgraph/+year+M|  & $0.2132 \times 831.76$(NNLO) & --- &988480
    \\[0.1cm]\hline
\end{tabular}
\end{adjustbox}

\end{table}

In the \ttjets samples, the NLO matrix element parton shower matching is varied by the damping 
parameter ($h_\text{damp}$). Additional $\ttjets$ samples are generated by varying 
$h_\text{damp}$ up and down and are used to observe the effect of $h_\text{damp}$. Similarly, 
$\ttjets$ samples where the renormalization and factorisation scales have been varied up 
and down are used to evaluate the uncertainties due to these scales. Also, alternate \ttjets 
samples with \mt = 171.5 and 173.5\GeV are considered to observe the effect of mass of the \PQt quark. 
All the additional \ttjets samples used for systematics study are shown in Table~\ref{tab:mcSampleSys}. 
\begin{table}
    \caption{ \ttjets  samples used for systematics study. Where
M = \/MINIAODSIM, year = run-v1, \\
run = { \em RunIISummer16MiniAODv2-PUMoriond17\_80X\_mcRun2\_asymptotic\_2016\_TrancheIV\_v6}. 
}
\label{tab:mcSampleSys}
\centering

\begin{adjustbox}{max width=\textwidth}
\begin{tabular}{ccc} \hline\hline
    {\bf{MC Dataset for Systematics}} & {\bf{$\sigma$}} (pb) & {\bf{Events}}\\\hline\hline
 \verb|/TT_TuneCUETP8M2T4up_13TeV-powheg-pythia8/+year+M|            &  730 & 29310620 \\[0.1cm]
 \verb|/TT_TuneCUETP8M2T4down_13TeV-powheg-pythia8/+year+M|         &  730 & 28354188 \\[0.1cm]
 \verb|/TT_TuneCUETP8M2T4_mtop1735_13TeV-powheg-pythia8/+year+M|    &  711 & 19419050 \\[0.1cm]
 \verb|/TT_TuneCUETP8M2T4_mtop1715_13TeV-powheg-pythia8/+year+M|    &  750 & 19578812 \\[0.1cm]
 \verb|/TT_hdampUP_TuneCUETP8M2T4_13TeV-powheg-pythia8/+year+M|     &  750 & 29689380 \\[0.1cm]
 \verb|/TT_hdampDOWN_TuneCUETP8M2T4_13TeV-powheg-pythia8/+year+M|   &  750 & 29117820
    \\[0.1cm]\hline
\end{tabular}
\end{adjustbox}

\end{table}

