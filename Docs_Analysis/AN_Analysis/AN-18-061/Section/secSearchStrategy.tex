 
The search for the charged Higgs boson in the $c\bar{s}$ channel at 13 
\TeV in the CMS experiment adopted a similar strategy as that of the 
previous analysis at 8 \TeV \cite{Khachatryan:2015uua}. An additional 
charm quark tagging have been further exploited to improve sensitivity. 
The invariant mass of the jets originating from charm and strange 
antiquark is taken as the observable for the search of charged Higgs, 
in the low mass region from 80 to 160 \GeV. In the absence of an 
excess in the observed data, a 95\% CL limit is put on the \brThb. The 
charm tagging is extensively used to improve this limit. As shown on 
the right side of the Figure~\ref{fig:feyn_diag_sig}, for the signal 
process, one \PQt quark decays to $H^+ b$ and the other one to 
$W^- \bar{b}$. The $W^+/H^+$ decays hadronically, whereas the $W^-$ 
decays leptonically. As a result, in the final states, there will be 
four jets (2 \PQb jets, 1 \PQc jet, 1 \PQs jet), one lepton (electron 
or muon, $\tau$ is not considered) and missing transverse energy 
attributed to neutrino. In this analysis, we assume that the \brHcs =100\%.
\begin{figure}
\centering
\includegraphics[width=0.75\textwidth]{Image/FeynDiag/feyn_diag_sig.pdf}
\caption{Production of \ttbar from gluon-gluon and quark-quark scattering. 
         The quark-scattering production process has a dominant contribution at 
         Tevatron energies whereas gluon-gluon scattering diagrams are dominant 
         at LHC energies ~\cite{Gerber:2014xea, Fiorini:2012fe}. The SM production
         of \ttbar is shown in (a), (b) and (c). The charged Higgs boson 
         production and its decay are shown in (d), (e) and (f).}
\label{fig:feyn_diag_sig}
\end{figure}

The standard model processes that give same final states (4 jets + 1 
lepton + missing energy) are considered as backgrounds for this 
analysis. The standard model \ttbar production is the most dominant, 
irreducible background process. As shown in the left side of the 
Figure~\ref{fig:feyn_diag_sig}, for SM \ttbar process, one \PQt quark 
decays to the $W^+$ and \PQb quark ($t\rightarrow W^+ b$) and the 
other decays to $W^- \bar{b}$ ($\bar{t}\rightarrow W^-\bar{b}$). The 
SM \ttbar contributes around 94\% of the total backgrounds. Other 
sub-dominant backgrounds that give rise to similar final states are 
single \PQt quark production, QCD multijet, \wjets, \dyjets, and 
vector boson fusion processes. The following background processes are 
considered for the search for charged Higgs. They are ordered in their 
significance of contribution.
\begin{enumerate}
\item $\textbf{SM \ttjets}$: Feynman diagrams 
	for \ttjets production are shown on the left hand side of Figure
	\ref{fig:feyn_diag_sig}. This is the most dominant background channel
	in the search for the signal search region (SR).

\item {\bf{Single \PQt}}: The single \PQt quark production process can also mimic the signal 
	topology. Three different ways, as shown in Figure~\ref{fig:feyn_diag_st}, 
	of production of single top quark considered in this analysis. It is produced through 
	s-channel, t-channel, and tW-channel. In the s-channel and t-channel the initial quarks 
	can be \PQu, \PQd, \PQc and \PQs (4-flavour scheme). However, in the tW-channel, 
	the initial quark is only \PQb quark (5-flavor scheme).
	\begin{figure}
	\begin{center}
	\includegraphics[width=0.75\textwidth]{Image/FeynDiag/feyn_diag_st.pdf}
	\caption{Representative Feynman diagrams for single \PQt quark production processes.}
	\label{fig:feyn_diag_st}
	\end{center}
	\end{figure}

\item {\bf{QCD multijet}}: The QCD multijet events contain only jets 
    at parton level. However, after event reconstruction, they can 
    still have leptons from misidentifications, and \MET due to poor 
    measurement of energy in the detector. Thus these events also 
    mimic the signal topology.

\item {\bf{\wjets}}: In this process, a \PW boson is produced in the 
    proton-proton collisions which subsequently decays leptonically 
    $(\PW^\pm \rightarrow l^+ \nu (l^-\bar{\nu}))$. The following 
    \wjets background process are considered in this analysis:
  	\begin{enumerate}
  	  \item $\PW + \text{jets  }$
  	  \item $\PW + \text{1 jet }$
  	  \item $\PW + \text{2 jets }$
  	  \item $\PW + \text{3 jets }$
  	  \item $\PW + \text{4 jets }$
  	\end{enumerate} 
	The Feynman diagram for these processes are shown in Figure~\ref{fig:feyn_diag_wjet}.
	\begin{figure}
	\begin{center}
	\includegraphics[width=0.75\textwidth]{Image/FeynDiag/feyn_diag_wjet.pdf}
	\caption{Representative Feynman diagrams for $\PW + \text{n jets}$ channel. The 
	\PW boson is produced by quark-quark and quark-gluon scattering 
	along with n jets (n = 0, 1, 2, 3, 4).}
	\label{fig:feyn_diag_wjet}
	\end{center}
	\end{figure}

\item ${\bf{Z/}}$ $\gamma$ ${\bf{+ jets}}$: The Drell-Yan processes in 
    which $Z/\gamma$ are produced along with jets, subsequently 
    decaying to two leptons $(Z/\gamma \rightarrow l^+ l^-)$, have 
    lepton and jets at parton level as shown in 
    Figure~\ref{fig:feyn_diag_dyjet}. However, after the 
    reconstruction, the \MET is also found in the events due to the 
    poor measurement of energy in the detector.
	\begin{enumerate}
  	\item $Z/\gamma +\text{ jets   }$
  	\item $Z/\gamma +\text{ 1 jet  }$
  	\item $Z/\gamma +\text{ 2 jets }$
  	\item $Z/\gamma +\text{ 3 jets }$
  	\item $Z/\gamma +\text{ 4 jets }$
  	\end{enumerate}
	\begin{figure}
	\begin{center}
	\includegraphics[width=0.75\textwidth]{Image/FeynDiag/feyn_diag_dyjet.pdf}
	\caption{Representative Feynman diagrams for $ Z/\gamma + \text{n jets}$ channel. 
	The $Z/\gamma$ is produced by quark-quark and quark-gluon scattering 
	along with n jets (n = 0, 1, 2, 3, 4).}
	\label{fig:feyn_diag_dyjet}
	\end{center}
	\end{figure}

\item {\bf{VV}}: Vector boson fusion processes are the smallest 
    background in the signal search region. The fusion happens via 
    tri-linear coupling between the $W^\pm$ and \PZ. The \PZ boson 
    further decays to $l^+l^-$. The VV process has three 
    sub-categories: WW, WZ, and ZZ. The vector Boson fusion process 
    is shown in Figure~\ref{fig:feyn_diag_vv}.
\begin{figure}
\begin{center}
\includegraphics[width=0.40\textwidth]{Image/FeynDiag/feyn_diag_vv.png}
\caption{Representative Feynman diagram for vector Boson fusion process.}
\label{fig:feyn_diag_vv}
\end{center}
\end{figure}
\end{enumerate}

