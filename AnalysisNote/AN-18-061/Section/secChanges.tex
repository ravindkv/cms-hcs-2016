
The analysis was started in early 2016 and published in late 2020. A
detailed timeline of the analysis is shown in the above figure. 
For every new version of the analysis note (AN), there were a few major 
changes w.r.t the previous version. In total there were 12 versions of
the AN. A detailed comparison of all major changes in the all versions
of the AN are listed in the following tables.

\begin{table}
\centering
\caption*{Authors and principal reviewers of the analysis note}
\begin{adjustbox}{max width=\textwidth}
\begin{tabular}{p{8cm}|p{8cm}}
\hline
{\textbf{Authors}} & {Shashi Dugad, Gouranga Kole, Gagan Mohanty, Arun Nayak, and Ravindra Kumar Verma}\\
\hline
{\textbf{Higgs-EXO conveners}} & {Abdollah Mohammadi and Anne-Marie}\\
\hline
{\textbf{Higgs-combine conveners}}     & {David Sperka, Andrea Marini, and Giacomo Ortona}\\
\hline
{\textbf{Higgs conveners}}     & { Giovanni Petrucciani, Roberto Salerno, and Maria Cepeda}\\
\hline
{\textbf{Analysis review committee}}     & { Meenakshi Narain (chair), Varun Sharma (interim chair), Paul Lujhan (language editor for paper), Sourabh Dube, and Anna Kropivnitskaya}\\
\hline
{\textbf{Introduction section}} & {Pankaj Jain (IIT Kanpur)}\\
\hline
\end{tabular}
\end{adjustbox}
\end{table}

\begin{table}
\centering
\caption*{Comparison of major changes in the AN from v11 and
v12.}
\begin{adjustbox}{max width=\textwidth}
\begin{tabular}{p{8cm}|p{8cm}}
\hline
{\bf{AN2018\_061\_v11 (23/09/2019)}} & {\bf{AN2018\_061\_v12 (29/12/2020)}}\\
\hline
The lepton data-to-simulation scale factors were 1 for \pt greater than the maximum \pt-range of 2D histograms & For $\pt > \pt^{max}$, scale factors are taken for $\pt^{max}$. Although they are close to 1\\
\hline
Used different seed for random number generation for Rochester correction for all systematics & Used same seed for all systematics \\
\hline
Used rc.kScaleFromGenMC() function with incorrect arguments for Rochester correction & Used rc.kScaleAndSmearMC() function with correct arguments\\
\hline
Description about the addition of exclusive and inclusive \PW/\PZ + jets samples was based on luminosity & Described them based on hepNUP variable \\
\hline
SF = $exp(0.0615 - 0.0005 \pt )$ for \PQt-quark \pt systematics with parton level \PQt-quark. Relative difference between the event yields without and applying SF$^2$ was considered as the systematics uncertainty& SF = $exp(0.076 - 0.00076\pt)$ with particle level \PQt-quark. Difference between nominal and SF applied. Rewrote \PQt-quark \pt reweighting section in more detail \\
\hline
QCD multijet sample \verb|QCD_Pt-15to20_EMEnriched| for electron channel & Removed it \\
\hline
Global tag \verb|80X_mcRun2_asymptotic| \verb|_2016_TrancheIV_v6| & \verb|80X_mcRun2_asymptotic| \verb|_2016_TrancheIV_v10| \\
\hline
- & Described the likelihood functions used for setting the upper limit \\
\hline
Only one table for showing the changes w.r.t. the current and previous versions of the analysis note & Kept all previous tables \\
\hline
- & Added timeline of the analysis \\
\hline
\textcolor{red}{Total 134 pages}                 & \textcolor{green}{Total 140 pages}\\\hline
\end{tabular}
\end{adjustbox}
\end{table}


\begin{table}
\centering
\caption*{Comparison of major changes in the AN from v10 and
v11.}
\begin{adjustbox}{max width=\textwidth}
\begin{tabular}{p{8cm}|p{8cm}}
\hline
{\bf{AN2018\_061\_v10 (09/09/2019) }} & {\bf{AN2018\_061\_v11 (23/09/2019) }}\\
\hline 
-              & Added correlation matrix of most significant fit parameters \\\hline
-              & Reproduced impact plots by using additional argument "- - cminDefaultMinimizerStrategy 0" in the command line\\\hline
-              & Used "newpage" latex command before a few sections\\\hline
-              & Rewrote and reshuffled a few sections for a better clarity \\\hline
Fit diagnostics in appendix     & It is in the main part  \\\hline
Impact plot in main part        & It is in the appendix  \\\hline
Useful commands in main part    & Put all commands in the appendix and referred them appropriately  \\\hline
\textcolor{red}{Total 127 pages}                 & \textcolor{green}{Total 134 pages}\\\hline
\end{tabular}
\end{adjustbox}
\end{table}

\begin{table}
\centering
\caption*{Comparison of major changes in the AN from v9 and
v10.}
\begin{adjustbox}{max width=\textwidth}
\begin{tabular}{p{8cm}|p{8cm}}
\hline
{\bf{AN2018\_061\_v9 (04/07/2019) }} & {\bf{AN2018\_061\_v10 (09/09/2019) }}\\
\hline 
Pre-fit yields and distributions  & Pre and post-fit yields and distributions\\\hline 
-              & Added $\mathcal{B}(t \rightarrow H^{+}b) = 0.065$ in plots\\\hline
-              & Added a plot showing final limits from 8 \TeV and 13 \TeV analysis \\\hline
\textcolor{red}{Total 124 pages}                 & \textcolor{green}{Total 127 pages}\\\hline
\end{tabular}
\end{adjustbox}
\end{table}

\begin{table}
\centering
\caption*{Comparison of major changes in the AN from v8 and
v9.}
\begin{adjustbox}{max width=\textwidth}
\begin{tabular}{p{8cm}|p{8cm}}
\hline
{\bf{AN2018\_061\_v8 (26/06/2019) }} & {\bf{AN2018\_061\_v9 (04/07/2019) }}\\
\hline 
Conservative styles for figures, plots, and tables  & Followed CMS guidelines for each of them\\\hline 
Blinded analysis              & Un-blinded analysis\\\hline
\textcolor{red}{Total 120 pages}                 &\textcolor{green}{Total 124 pages}\\\hline
\end{tabular}
\end{adjustbox}
\end{table}

\begin{table}
\centering
\caption*{Comparison of major changes in the AN from v7 and
v8.}
\begin{adjustbox}{max width=\textwidth}
\begin{tabular}{p{8cm}|p{8cm}}
\hline
{\bf{AN2018\_061\_v7 (17/04/2019) }} & {\bf{AN2018\_061\_v8 (26/06/2019) }}\\
\hline 
$M_{H^\pm}$, $M_{jj}$           & $m_{H^+}$, $m_{jj}$ \\\hline 
Considered systematic uncertainty in the QCD estimation as change in the transfer 
scale factor & Considered it as a change in the yield \\\hline
-              & Added uncertainty table for exclusive charm categories\\\hline
\textcolor{red}{Total 118 pages}                 & \textcolor{green}{Total 120 pages}\\\hline
\end{tabular}
\end{adjustbox}
\end{table}

\begin{table}
\centering
\caption*{Comparison of major changes in the AN from v6 and
v7.}
\begin{adjustbox}{max width=\textwidth}
\begin{tabular}{p{8cm}|p{8cm}}
\hline
{\bf{AN2018\_061\_v6 (01/02/2019) }} & {\bf{AN2018\_061\_v7 (17/04/2019) }}\\
\hline 
Muon: \verb|HLT_IsoMu24*|       & \verb|HLT_IsoMu24*| OR \verb|HLT_IsoTkMu*|\\\hline
Muon: $\eta < 2.1$              & $\eta  < 2.4$ \\\hline 
Electron: $d0 <0.05$, $dz < 0.20$ & $ d0<0.05$, $dz<0.10$ (in barrel), and $d0<0.10$, $dz<0.20$ (in endcap)\\\hline
Electron: relative isolation $< 0.08$      & Relative isolation $< 0.0821$ for barrel and $< 0.0695$ for endcap\\\hline 
-                               & For jets: charged hadron multiplicity $> 0$, number of constituent $> 1$, charged hadron electromagnetic energy fraction $<0.99$\\\hline
\ptmiss filters not applied     & All filters have been applied \\\hline
-                               & Scaled signal events by the maximum observed upper limit at 8 \TeV in the plots.\\\hline
-                               & Added table for event yields from exclusive charm categories.\\\hline
A brief description of the kinematic fitting  & Described it in detail \\\hline
-                               & Added reconstruction, identification, and trigger efficiency plots for electron and muon\\\hline
-                               & Added tables for \PQb/\PQc tagging efficiency\\\hline
-                               & Updated all the tables and plots for a better visibility\\\hline
Unblinded analysis              & Blinded analysis\\\hline
\textcolor{red}{Total 110 pages}                 & \textcolor{green}{Total 118 pages}\\\hline
\end{tabular}
\end{adjustbox}
\end{table}


\begin{table}
\centering
\caption*{Comparison of major changes in the AN from v5 and
v6.}
\begin{adjustbox}{max width=\textwidth}
\begin{tabular}{p{8cm}|p{8cm}}
\hline
{\bf{AN2018\_061\_v5 (11/01/2019) }} & {\bf{AN2018\_061\_v6 (01/02/2019) }}\\
\hline
Blinded analysis i.e. the observed data was shown only in the control regions and the exclusion limits were computed using MC simulation& Un-blinded analysis i.e. the observed data was also used in the signal region and limit computation\\
\hline
- & Added goodness-of-fit test\\
\hline
The systematic uncertainties corresponding to the renormalization and factorisation scale and damping parameters  were profiled as "shape"& These are profiled as "lnN" as the ratio of these with nominal histograms is consistent with a flat line\\
\hline
\textcolor{red}{Total 109 pages}                 & \textcolor{green}{Total 110 pages}\\\hline
\end{tabular}
\end{adjustbox}
\end{table}

\begin{table}
\centering
\caption*{Comparison of major changes in the AN from v4 and
v5.}
\begin{adjustbox}{max width=\textwidth}
\begin{tabular}{p{9cm}|p{9cm}}
\hline
{\bf{AN2018\_061\_v4 (14/05/2018)}} & {\bf{AN2018\_061\_v5 (11/01/2019) }}\\
\hline
${\pt}_{\mu} > 25$ GeV & ${\pt}_{\mu} > 26$ GeV  \\
\hline
Applied \PQb/\PQc tag data-to-simulation scale factors using the promote-demote method & Applied these scale factors using event-by-event reweigting\\
\hline
- & Added TProfile plots between \ptmiss and $I_{rel}$ for simulated QCD multijet sample. Removed few columns (after applying various scale factors) from cutflow tables. Considered one extra nuisance parameter as a systematic uncertainty for exclusive charm categories. Applied a proper correlation between \PQb/\PQc-tagged scale factors. Added signal to background ratio plot from exclusive charm categories for all masses of the charged Higgs. Added a section on the profiling of nuisance parameters in the likelihood. For comparison, added plots showing the ratio between base and up or down histograms for the different systematic uncertainties\\
\hline
Considered one nuisance parameter in each bin of only \ttbar, single \PQt, \PW+jets, and
charged Higgs signal samples, manually, for bin-by-bin uncertainties & Assigned one nuisance parameter in each bin in the sum of all background and signal processes using autoMCStats tool which is based on Barlow-Beeston-lite approach. \textcolor{red}{After using autoMCStats, event categorisation in bins of \pt of \PQb jets does not improve the expected limit}. Hence, moved \PQb jet \pt binning section in the appendix \\
\hline
Systematic uncertainty in the luminosity measurement = 2.7\% and lepton selection = 3.1\% & It is 2.5\% and 3.0\%, respectively \\
\hline
Various systematic uncertainties such as those from \PQb/\PQc tagging, jet energy corrections (jet energy scale and resolution), pileup reweighting, \PQt-quark \pt reweighting, and measurement in the \PQt-quark mass were profiled as "shape" in the data card for limit computation & Now these profiled as "lnN" \\ 
\hline
NPs from statistical uncertainty in the total number of events from each simulated sample were included in the data card & These NPs are removed from the data card, as they are internally included from autoMCStats tool \\
\hline
The NPs were named following the conventions of 8 \TeV analysis. & They are renamed using recommended conventions at 13 \TeV\\
\hline
- & Added plots for fit diagnostics, the variation of likelihood as function of individual NP, and showing the impacts of NPs on the parameter of interest \\
\hline
- & Updated all tables with latest results, proper fonts etc\\
\hline
The stacked and ratio plots were in a different pads & They are in the same pad \\
\hline
- & Added plots showing the observed data and simulated events from a control region ($\ptmiss < $) 20 GeV\\
\hline
- & Added a section on the background contribution from all SM Higgs production processes. Included limits for the charged Higgs mass of 80 GeV\\
\hline
\textcolor{red}{Total 81 pages}                 & \textcolor{green}{Total 109 pages}\\\hline
\end{tabular}
\end{adjustbox}
\end{table}


\begin{table}
\centering
\caption*{Comparison of major changes in the AN from v3 and
v4.}
\begin{adjustbox}{max width=\textwidth}
\begin{tabular}{p{8cm}|p{8cm}}
\hline
{\bf{AN2018\_061\_v3 (04/05/2018) }} & {\bf{AN2018\_061\_v4 (14/05/2018) }}\\
\hline
- & Added columns in table 17 and 18 for the systematic uncertainties corresponding to the mis-identified \PQb and \PQc jet tagging\\
\hline
\textcolor{red}{Total 81 pages}                 & \textcolor{green}{Total 81 pages}\\\hline
\end{tabular}
\end{adjustbox}
\end{table}


\begin{table}
\centering
\caption*{Comparison of major changes in the AN from v2 and
v3.}
\begin{adjustbox}{max width=\textwidth}
\begin{tabular}{p{8cm}|p{8cm}}
\hline
{\bf{AN2018\_061\_v2 (06/04/2018) }} & {\bf{AN2018\_061\_v3 (04/05/2018) }}\\
\hline
Integrated luminosity = 35.50 \fbinv & 35.9 \fbinv \\
\hline
Muon isolated data-to-simulation scale factors were applied to all muons (isolated and non-isolated) & Only applied to isolated muons\\
\hline
Kinematic cuts (Table-4) + medium muon ID for second muon to be vetoed &  Only Kinematic cuts (Table-4) for the second muon to be vetoed with no requirement on tracker related or $\chi^2$ cuts\\
\hline
Electron relative isolation cut was, by mistake, not applied inside the code on the second electron to be vetoed (although it is shown in Table-5) &  Now applied \\ 
\hline
While evaluating data-to-simulation normalization, the cross-section for the charged Higgs sample was taken to be $ 0.32 \times \sigma_\ttbar$ & Now, it is taken to be $ 0.2132 \times \sigma_\ttbar$ \\
\hline
A fixed 3.3\% pileup uncertainty was considered for all simulated samples & The minimum bias cross section is varied by 4.7\% up and down and the corresponding pileup weights are applied. The pileup uncertainty is determined from the change w.r.t to base value \\
\hline
The \PQb and \PQc-tagged data-to-simulated scale factors were applied for the 
\ttbar, charged Higgs signal, and single top production processes. For the rest 
simulated samples such as as \PW+jets, \PZ+jets, VV processes, the corresponding 
mis-tagged scale factors were applied  & Now the \PQb and \PQc-tagged scale factors 
are applied for all simulated samples \\ 
\hline
\textcolor{red}{Total 79 pages}                 & \textcolor{green}{Total 81 pages}\\\hline
\end{tabular}
\end{adjustbox}
\end{table}

\begin{table}
\centering
\caption*{Comparison of major changes in the AN from v1 and
v2.}
\begin{adjustbox}{max width=\textwidth}
\begin{tabular}{p{8cm}|p{8cm}}
\hline
{\bf{AN2018\_061\_v1 (12/03/2018)}} & {\bf{AN2018\_061\_v2 (06/04/2018) }}\\
\hline
Being the first version of the AN, the documentation was too preliminary & Following the suggestions from authors, updated whole AN to be reviewed by the Higgs-EXO conveners\\
\hline
\textcolor{red}{Total 57 pages}                 & \textcolor{green}{Total 79 pages}\\\hline
\end{tabular}
\end{adjustbox}
\end{table}
