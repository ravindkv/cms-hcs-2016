
In this section, we describe the process by which charged Higgs could be produced. It is
also important to understand all the standard model background process that would mimic
the signal process. The contribution of each of these known background processes is estimated
using MC simulation. The signal MC distribution is also simulated using a similar technique.
The dijet mass distribution of highest \pt jets (non b-jets) is used to search for charged 
Higgs. The dominant Feynman diagram where charged Higgs Bosons are produced from the semi-leptonic 
decay of \ttbar are shown in Figure~\ref{fig:feyn_diag_sig}d, ~\ref{fig:feyn_diag_sig}e, 
~\ref{fig:feyn_diag_sig}f). As shown in this figure, the charged Higgs is assumed to decay into 
$c\bar{s}$ only. As a result, in the final states, there will be 4 jets (2 b-jets, 1 c-jet, 1
s-jet), 1-lepton (electron or muon, $\tau$ is not considered in this analysis) and missing energy.
Standard Model process that gives the same final states (4 jets + 1 lepton + missing energy) are considered as 
background channel for this analysis. The following background processes are considered in this
analysis. They are ordered in their significance of contribution.
 \begin{enumerate}
%########### t tbar #############
     \item $\textbf{t}\bar{\textbf{t}} + \textbf{jets}$: Feynman diagrams for \ttjets production are shown in
       Figure~\ref{fig:feyn_diag_sig}a, ~\ref{fig:feyn_diag_sig}b, ~\ref{fig:feyn_diag_sig}c. This 
      is the most dominant background channel in the search for the signal region (SR).
%########### QCD #############
   \item {\bf{QCD}}: The QCD events contain only jets at parton level. However, after
       event reconstruction, QCD events can still have leptons from the misidentifications,
       and \MET due to poor measurement of energy in the detector. Thus the QCD events also
       mimic the signal topology.
    
%########### Single Top #############
   \item {\bf{Single t}}: The single top production can also mimic the signal topology. Three 
       different ways as given below (Figure~\ref{fig:feyn_diag_st}) of production of single top 
       quark considered in this analysis.
   \begin{enumerate}
     \item ST\_s-channel
     \item ST\_t-channel
     \item ST\_tW
   \end{enumerate}

%########### W + jets #############
   \item {\bf{W + jets}}: In this process, a W Boson is produced in the p p collision 
       which subsequently decays leptonically. Following W + jets background process 
       are considered in this analysis:
   \begin{enumerate}
     \item $W + jets   (W^\pm \rightarrow l^+ \nu (l^-\bar{\nu}))$
     \item $W + 1 jet  (W^\pm \rightarrow l^+ \nu (l^-\bar{\nu}))$
     \item $W + 2 jets (W^\pm \rightarrow l^+ \nu (l^-\bar{\nu}))$
     \item $W + 3 jets (W^\pm \rightarrow l^+ \nu (l^-\bar{\nu}))$
     \item $W + 4 jets (W^\pm \rightarrow l^+ \nu (l^-\bar{\nu}))$
   \end{enumerate}
   The Feynman diagram for these processes are shown in Figure~\ref{fig:feyn_diag_wjet}.

 %########### DY + jets #############
   \item ${\bf{Z/}}$ $\gamma$ ${\bf{+ jets}}$: The Drell-Yan process in which $Z/\gamma$
       are produced with jets have \ljets at parton level as shown in
       Figure~\ref{fig:feyn_diag_dyjet}.
       However, after the reconstruction, the \MET is also found in the events due to the poor
       measurement of energy in the detector.
   \begin{enumerate}
     \item $Z/\gamma + jets   (Z/\gamma \rightarrow l^+ l^-)$
     \item $Z/\gamma + 1 jet  (Z/\gamma \rightarrow l^+ l^-)$
     \item $Z/\gamma + 2 jets (Z/\gamma \rightarrow l^+ l^-)$
     \item $Z/\gamma + 3 jets (Z/\gamma \rightarrow l^+ l^-)$
     \item $Z/\gamma + 4 jets (Z/\gamma \rightarrow l^+ l^-)$
   \end{enumerate}
 
   %########### VV #############
   \item {\bf{VV}}: Vector Boson fusion process as shown in Figure~\ref{fig:feyn_diag_vv}
       are the smallest background in the SR. The VV process has three sub-categories:
   \begin{enumerate}
     \item WW
     \item WZ
     \item WZ
   \end{enumerate}
 \end{enumerate}
