There are two b-jets in the final state as shown in Figure~\ref{fig:feyn_diag_sig} for charged Higgs
signal as well as SM \ttbar+jets background.
Accurate identification of these b-jets will substantially reduce the SM backgrounds where there are no b-quarks at parton level such as $Z/\gamma + jets$, $VV$, $W + jets$ etc.
The {\em combined secondary vertex (CSVv2)} method~\cite{Chatrchyan:2012jua} is used to tag a b-jet.
The main idea behind the method is that b-hadrons produced from the b-quark (originating from the PV) 
have a relatively larger lifetime due to which they can travel measurable distance w.r.t. primary
vertex in the transverse plane before decaying further (Figure~\ref{fig:bCSV}). The position
resolution of vertex in the transverse plane is about few $\mu$m. Therefore, a secondary vertex displaced by few hundred microns w.r.t. primary vertex can be identified and reconstructed. 
The transverse distance of tracks from the PV is called impact parameter (IP).
To calculate the b-discriminator, first tracks of reasonable quality are selected, following which
secondary vertex is reconstructed. Finally, multi-layer perceptrons~\cite{CMS-PAS-BTV-15-001} is used
to accept or reject the secondary vertex, discriminator value of the jet.
\begin{figure}
\centering
\includegraphics[width=0.50\linewidth]{Image/bCSV.pdf}
\caption{ Primary and secondary vertex with tracks and impact parameter. IP $> 0$ or $< 0$ if tracks
are upstream or downstream along the jet direction. This figure is taken from Ref.~\cite{Ferro:2012tg}.} 
\label{fig:bCSV}
\end{figure}

\subsection{Track Selection}
\label{ss:Track Selection}
The tracks are selected with the criteria listed in Table~\ref{tab:track_sel}.
These requirements ensure the tracks are closer to the PV and not coming from pileup vertices.
After track selection, the SV is reconstructed.
\begin{table}
  \caption{Track selection criteria. PCA stands for the point of closest approach.}
 \begin{center}
 \begin{tabular}{cc}\hline\hline
 Variable & Selection \\ \hline\hline
 $\pt$ associated with tracks & $>0.8$ GeV \\
 number of hits associated with tracks in the tracker & $>8$ \\
 longitudinal component of IP & $< 0.3$~\unit{cm}\\
 distance between PCA of tracks from the PV and jet-axis & $<0.07$~\unit{cm}\\
 distance between PCA of tracks from the jet-axis and PV & $< 5$~\unit{cm}\\
 the displaced track should have & IP $>50$~\unit{$\mu$m} and $\frac{\rm IP}{\sigma_{\rm IP}}>1.2$\\\hline
 \end{tabular}
 \end{center}
 \label{tab:track_sel}
 \end{table}

\subsection{Reconstruction of Secondary Vertex}
For Run-I, the {\em combined secondary vertex (CSV)} method used the adaptive vertex
reconstruction algorithm~\cite{Waltenberger:1166320}, which is based on adaptive vertex
fitter~\cite{Fruhwirth:1027031} to reconstruct the SV.
On the other hand, in case of Run-II, the CSVv2 uses the inclusive vertex finder (IVF)
algorithm~\cite{Khachatryan:2011wq} for the same purpose.
The collection of selected tracks are used as inputs to the IVF. The displaced tracks are used as seed.
The cluster of tracks is formed from seed track depending on the angles and distance between them.
The adaptive vertex fitter is used to fit the clusters.
The selection listed in Table~\ref{tab:sv_sel} are applied for the SV.
\begin{table}
  \caption{Secondary vertex selection criteria. $d_{\rm PV, SV}^{\rm trans}$ is the 2D flight
  distance or the distance between a primary and secondary vertex in the transverse plane.}
 \begin{center}
 \begin{tabular}{cc}\hline\hline
 Variable & selection \\ \hline\hline
     number of tracks associated with the SV & $>$ 2 \\
     $d_{\rm PV, SV}^{\rm trans}$ & $>$ 0.1~\unit{mm} \\
     $d_{\rm PV, SV}^{\rm trans}$ & $<$ 2.5~\unit{cm} \\
     mass associated with the SV & $<$ 6.5 GeV\\
     $\Delta R$ (jet axis, secondary flight direction) & $<$ 0.4\\ \hline
 \end{tabular}
 \end{center}
 \label{tab:sv_sel}
 \end{table}

\subsection{Multi-layer Perceptron Training}
Three types of discriminating variable, as shown below, are used in multi-layer perceptron training to identify the secondary vertex.
\begin{itemize}
    \item category-I: jet with $N_{\rm SV} > 1$,
    \item category-II: jet with pseudo-vertex~\cite{CMS-PAS-BTV-15-001}, and
    \item category-III: jet with no SV or no pseudo-vertex.
\end{itemize}
The discriminating variables from the vertex-categories are combined using a likelihood ratio,
which gives the b-discriminator value.
The b-discriminator value for the \mujets and \ejets channel is shown in
Figure~\ref{fig:pfCISV_lepBTag}.
There are three official b-tag working points: {\em loose}, {\em medium}, and {\em tight} with
b-tag efficiency [defined in Eq.~(\ref{eq:btag_eff})] 81\%, 63\%, and 41\%, respectively.
The corresponding probability of a light jet being misidentified as a b-jet is 10\%, 1\%, and 0.1\%.
In this analysis, we use the medium working point i.e., b-discriminator $>$ 0.8484 for b-tagging.
The efficiency of all b-taggers for b, c, and light-quarks are shown in Table (\ref{tab:bTagEff}). 
\begin{table}
\begin{center}
\caption{The efficiency of loose, medium, and tight b-tag working points for different quark-flavor
    of jets \cite{Sirunyan:2017ezt}. These efficiencies are calculated from
\ttbar events with jet \pt $>$ 20 GeV.}
\begin{tabular}{cccccc}
\hline
\hline
WP & $\epsilon^b$ (\%) & $\epsilon^c$ (\%) & $\epsilon^{udsg}$ (\%)& CSVv2 \\ \hline\hline
b-tagger L & 81 & 37 & 8.9 & $>$ 0.5426  \\
b-tagger M & 63 & 12 & 0.9 & $>$ 0.8484   \\
b-tagger T & 41 & 2.2& 0.1 & $>$ 0.9535   \\
\hline
\end{tabular}
\label{tab:bTagEff}
\end{center}
\end{table}

\begin{center}
\begin{figure}
\subfigure[b-jet discriminator for the \mujets channel]{\includegraphics[width=0.50\linewidth]{Image/Muon/BTag/pfCISV_muBTag.pdf}}
\subfigure[b-jet discriminator for the \ejets channel]{\includegraphics[width=0.50\linewidth]{Image/Electron/BTag/pfCISV_eleBTag.pdf}}
\caption{The b-discriminator distributions of all jets, after $N_{jets} \geq 4$ selection as described in Sec.~\ref{s:secEvtSel}, obtained using CSVv2 method for the \mujets and \ejets channel.}
\label{fig:pfCISV_lepBTag}
\end{figure}
\end{center}

