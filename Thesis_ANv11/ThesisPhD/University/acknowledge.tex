\chapter{Acknowledgments}

\lettrine[lines=2, findent=3pt, nindent=0pt]{A}{} dream came true for me. A few 
years back in 2008, when the LHC was about to collide the two oppositely
moving proton for the first time in history, there was an avalanche of news
around every corner of the world about it. There were misconceptions that
during the collision, a microscopic black hole will be created which will start 
sucking the matter around it and eventually the whole planet earth after some time.
As a student of physical science in 12th standard, I was very curious to know 
more about the LHC in those days. There was one edition of \textit{Vigyan Pragati},
a science magazine in Hindi language, about the physics activities happening at the
LHC. After reading that, I became even more curious about the LHC. However, at 
that time, I had never imagined that in the future, I would end up doing a Ph.D. 
in the LHC! The journey from one of the villages in the north India to one of the world's most 
sophisticated experimental setups in Switzerland has been filled with a 
lot of ups and downs. There have been many people who helped me directly or 
indirectly during this journey. Here I want to acknowledge a few people for their support during this long journey.

\begin{itemize} [leftmargin=*]
\item \textbf{12th}:
The first step of this journey started in 12th standard in my school (\textit{Pioneer Montessori Inter College}) where the foundation was laid. My ideal teacher (\textit{Mahendra Pratap Singh}) taught us physics and mathematics. He is not only a great teacher but also a great motivator. I am fortunate enough to have many school friends \textit{Siddhant, Ashutosh, Mayank, Shahbaz, Sandeep, Prashant, Shailesh}, etc, who made my school life memorable.

\item \textbf{B.Sc.}:
My best friend \textit{Siddhant} guided me and my late Fufa 
(\textit{Santsaran}) financially supported me to get admission in \textit{Kamla 
Nehru Institute of Physical and Social Sciences} into $B.Sc.$ program. I got the opportunity to learn college level physics and mathematics there. I am indebted to my professors - \textit{
Yogendra Bahadur Singh, Pankaj Singh, Prashant Singh, Lalit Kumar Divedi, Jaysnath 
Mishra} who cultivated my understanding of physics and mathematics. During these 
years, I made a lot of friends - \textit{Ravi, Prem, Anil,
Zeeshan, Chand, Sagar, Deepoo, Syamoo, Arunesh, Mayank, and Satyajit} and they made 
the college life super enjoyable. I sincerely thank them for being with me when
I needed them the most.

\item \textbf{M.Sc.}:
I was fortunate enough to get admission in the M.Sc.-Ph.D. dual degree 
program for physics in \textit{Indian Institute of Technology, Kanpur} (IITK) 
which is one of the best institutes for higher education in India. During the
M.Sc. program, I got the opportunity to learn from the advanced physics courses taught by the leading physicist at IITK. I enjoyed the courses taught by \textit{
Deshdeep Sahdev, Manoj Kumar Harbola, Pankaj Jain, Debashish Chowdhury, and Joydeep Chakraborty}. During the M.Sc. program, I got the opportunity to attend classes with the best minds of India. I am thankful to my friends \textit{Vimalesh, Sangha Mitra, Chitrasen, Dhananjay, and Santosh} for making
IITK life memorable. I enjoyed the discussions/debates with my best friend
Vimalesh on a wide range of issues ranging from science, society, politics, 
philosophy, sports, etc. We spent many hours on these issues and ended most of
the time without reaching to any logical conclusion. 

\item \textbf{Ph.D.}: 
In the first two years of my Ph.D. program, I pursued my research in the area
of theoretical high energy physics working on the dark matter, very special
relativity (VSR), and top-quark threshold effect. I am very grateful to my senior
\textit{Alekha} for helping and correcting me at many points. I also thank 
\textit{Prasenjit} and \textit{Subhadip} for many theoretical discussions. In my 
third year of Ph.D., I started my research in the experimental high energy 
physics and joined Prof. Shashi Dugad at \textit{Tata Institute of Fundamental 
Research} as a junior research fellow in the fourth year. During the transition from theoretical 
to the experimental field, I met one of the most beautiful girl and my one of the 
best friends, \textit{Swyamsree Patra}. On academic leave from IITK, I deputed to 
TIFR in May 2016 to explore the LHC world. I will always be indebted to my 
supervisors \textit{Pankaj Jain} and \textit{Shashi Dugad} for giving me such an 
excellent opportunity.

It is beyond words to describe the experience that I got while working at TIFR
and CERN. I am very thankful to \textit{Gouranga} and \textit{Arun Nayak} for 
guiding me at every technical detail. Having a physics discussion with 
\textit{Gagan} and \textit{Shashi} was very useful. I am also grateful to 
\textit{Giovanni Franzoni, Giacomo Govi, Arun Kumar, Luca}, and \textit{Tongguang}
for giving me the opportunity to work as a level-3 convener in the AlCaDB 
(alignment calibration and database) group. I also benefited from the discussion 
with Higgs conveners (\textit{Anne-Marie, Abdollah, Giacomo Ortona, Andra, Giovanni 
Petruciani, and Maria Cepeda}). I will always be grateful to \textit{Bajarang} who was senior to
me but always treated me like a younger brother. He helped me a lot on personal 
and professional front in the initial days at TIFR. A very special thank to him. 
During this period, I got a few amazing friends - \textit{
Muzamil, Bibhu, Raghu, Irfan, Sushil, Amey, Manish, Alibordi, Shalini, Akshansh, Pritam, Ashish, etc}. 
I also would like to thank \textit{Brij} and \textit{Puneet} for being available to fix 
technical issues related to computing infrastructures, specifically providing 
condor facility for running parallel jobs.
\end{itemize}

My special thanks to Prof. Pankaj Jain who is one my ideals and a living legend. 
Working under him in the past 6 years was quite rewarding on personal as well 
as professional fronts. I would also thank Prof. Shashi Dugad with my folded hand 
for always with me during the last 3 years. For him, I would use the phrase 
\textit{A silent sea never made a skilled sailor}. After working under him, I feel 
trained enough to sail in any \textit{sea}.

I am also grateful to the administration members at Department of Physics 
(IITK) and Department of High Energy Physics (TIFR) for assisting me 
with various paperworks. I am also thankful to MHRD and DAE for financial 
support during my PhD at IITK and TIFR respectively.

Finally, I would like to thank my parents for giving me the freedom to choose the 
academic carrier as per my wish. I also thank the other members of my family and 
village - \textit{Ramesh, Sachin, Manju, Parvind, Dipak, Akash, Vikash, 
Ankit, Surendra} for being with me since my childhood. I lost my Fufa (
\textit{Santsaran}) in a car accident on February 10, 2010. He was a father-figure for me. He was one of the people who motivated and helped me when I needed the most. Had he been alive, he would have been very happy to see the fruits of the tree, he had planted a decade ago. At the end, I would like to thank my fiance (\textit {Swayamsree Patra}) for taking my care and loving me.

\vspace{20pt}
\begin{flushright}
	\textsc{Ravindra Kumar Verma}\\\textsc{IIT Kanpur}\\
	\textsc{Uttar Pradesh}, 208016
\end{flushright}
