
In order to explain some failures of the standard model, the 2HDM postulated 
existence of an additional Higgs doublet. It leads to two charged and three neutral
physical scalar particles. A charged Higgs boson 
decays to various channels depending on the parameter space corresponding to its mass and 
$\tan\beta$. A search for the charged Higgs in the $H^+ \to c\bar{s}$ channel has been 
performed in this thesis with the data collected by the CMS experiment at the center-of-mass 
energy of 13 \TeV with an integrated luminosity of 35.9\fbinv. The observed data and 
standard model predictions are in agreement within the statistical and systematic uncertainties. 
That is, no signal has been observed in the data for the mass of charged Higgs in the range
80 to 160 \GeV. In the absence of an excess, a 95\% CL upper limit has been put on \brThb, 
assuming \brHcs = 100\%. The kinematic fitting and a new technique which uses \PQc tagging has 
been intensively exploited to improve the limits. The final limits from combined charm categories 
are shown in Tables \ref{tab:limit_muon}, \ref{tab:limit_electron}, and \ref{tab:limit_lepton}. 
The observed limits are in the range 0.29--2.12\%, 0.27--3.29\%, and 0.20--1.65\% for 
\mujets, \ejets, and \ljets channel, respectively. The observed limits at 13 \TeV are better 
by an average factor of 7.6 as compared to the earlier CMS results at 8 \TeV~\cite{Khachatryan:2015uua}.
The improvement in the limit is partly due to higher luminosity and partly due the new technique 
used in this thesis. 

As a continuation of the analysis presented in this thesis, here are a few points that 
can/will be done in the near future: 
\begin{itemize}[leftmargin=*]
\item As of now, the search for charged Higgs in \brHcs channel at 13 \TeV has not been made public 
from the ATLAS experiment. It would very interesting to compare our results with that from the ATLAS 
once they become available. Currently, we have analysed only a fraction of data recorded during
Run-II (year 2016, 2017, and 2018). Analysing proton-proton collision data recorded by the 
CMS experiment from full Run-II, which correspond to an integrated luminosity of 160\fbinv, 
would put tighter constraints on \brThb. The new limit should be further constrained by about a 
factor of $\sqrt{160/36}\approx 2$ if one goes by luminosity only. However, advanced
techniques such as multivariate analysis, deep \PQb/\PQc tagging, etc, will further constrain
the upper limits. Finally, combining the final limits with 160\fbinv data from the CMS and 
ATLAS experiments will put even tighter constraints on \brThb. 

\item Since we already know the allowed region of \brThb at 13 \TeV with 35.9\fbinv data, we can
use these experimental results to see what value of $\tan\beta$ is allowed for a different 
mass of the charged Higgs. Note that for the signal hypothesis, during the generation of 
simulated signal samples, we have not put any constraint on the value of $\tan\beta$. We have
kept it floating so that we can restrict it once we know the limit on \brThb. However, in the
context of the 2HDM,  we will have to restrict ourselves to those Types where \brHcs is close to
100\%. For this, we have to interpret the current experimental results in the Type-II for lower
values of $\tan\beta$. Using the formula for \brThb, given in Equation (\ref{eq:brThb}), we can impose
a limit on $\tan\beta$. In fact, such
a phenomenological study has already been carried out using the $\tau\nu$, $t\bar{b}$ channels
\cite{Sanyal:2019xcp}, where they have used \rm{HDECAY} package~\cite{Djouadi:1997yw} to generate 
theoretical values of the branching fractions for different charged Higgs masses. We hope a similar 
study can/will be carried out for the $c\bar{s}$ channel also. 

\item So far in the experimental analysis we have considered \brHcs = 100\% and put upper limits
on \brThb at 95\% CL. However, we can relax this assumption and do a similar analysis where
we will put limits on $x$ = \brThb $\times$ \brHcs, where \brHcs $\leq$ 1. For this, we will have 
to modify the Equation (\ref{eq:deltaN}) so that the signal and background process are scaled 
appropriately under the new parameter of interest. The relaxation of the
present assumption (\brHcs = 100\%) would be interesting for the phenomenology of charged Higgs 
in models like the 2HDM and MSSM. As the Higgs sector of the MSSM corresponds to the Type-II of 
the 2HDM, and as shown in Figure~\ref{fig:brHqq}, the \brHcs is dominant (\ie close to 100\%) 
only when $\tan\beta < 1$. In Type-X also, we see a similar trend. For the other two types 
of 2HDM (Type-I and Y), the \brHcs could not be realised for any choice of $\tan\beta$. However,
relaxing the assumption \brHcs = 100\% would definitely help in imposing limits on 
$\mHp$--$\tan\beta$ parameter space for any choice of $\tan\beta$ for all types of 2HDM and MSSM.

\item Currently, different channels such as $\tau^+\nu_\tau$, $c\bar{s}$, $t\bar{b}$, $c\bar{b}$, etc 
of the charged Higgs decays are examined separately. It would be very interesting if all results from
all the channels are combined and a final limit on \brThb is set under the assumption that the charged
Higgs decays to everything ($H^+ \to q\bar{q} \rm{ and } l^+\nu_l$).  
\end{itemize}

\begin{table}
\caption{95\% \CL exclusion limits in \% for the \mujets from combined \PQc tagging categories.}
\label{tab:limit_muon}
\begin{center}
\begin{tabular}{ccccccc}
\hline
\hline
\multicolumn{1}{c}{$\mHp$ (GeV)} & \multicolumn{5}{c}{{Expected}} & \multicolumn{1}{c}{{Observed}} \\
&\multicolumn{1}{c}{$-2\sigma$} &\multicolumn{1}{c}{$-1\sigma$} &\multicolumn{1}{c}{median} & \multicolumn{1}{c}{$+1\sigma$} & \multicolumn{1}{c}{$+2\sigma$}&\\ \hline
\hline
80  & 1.62 & 2.18 & 3.06 & 4.31 & 5.85 & 2.12\\
90  & 0.72 & 0.96 & 1.34 & 1.87 & 2.49 & 0.79\\
100 & 0.36 & 0.49 & 0.68 & 0.94 & 1.25 & 0.38\\
120 & 0.25 & 0.33 & 0.46 & 0.64 & 0.86 & 0.29\\
140 & 0.22 & 0.30 & 0.41 & 0.57 & 0.75 & 0.33\\
150 & 0.21 & 0.28 & 0.39 & 0.54 & 0.72 & 0.37\\
155 & 0.22 & 0.30 & 0.41 & 0.58 & 0.77 & 0.45\\
160 & 0.23 & 0.31 & 0.44 & 0.62 & 0.84 & 0.44\\
\hline
\end{tabular}
\end{center}
\end{table}

\begin{table}
\caption{95\% \CL exclusion limits in \% for the \ejets from combined \PQc tagging categories.}
\label{tab:limit_electron}
\begin{center}
\begin{tabular}{ccccccc}
\hline
\hline
\multicolumn{1}{c}{$\mHp$ (GeV)} & \multicolumn{5}{c}{{Expected}} & \multicolumn{1}{c}{{Observed}} \\
&\multicolumn{1}{c}{$-2\sigma$} &\multicolumn{1}{c}{$-1\sigma$} &\multicolumn{1}{c}{median} & \multicolumn{1}{c}{$+1\sigma$} & \multicolumn{1}{c}{$+2\sigma$}&\\ \hline
\hline
80  & 1.97 & 2.63 & 3.66 & 5.14 & 6.89 & 3.29\\
90  & 0.80 & 1.07 & 1.48 & 2.07 & 2.77 & 1.45\\
100 & 0.43 & 0.56 & 0.78 & 1.09 & 1.45 & 0.61\\
120 & 0.29 & 0.38 & 0.53 & 0.73 & 0.97 & 0.53\\
140 & 0.24 & 0.32 & 0.45 & 0.62 & 0.83 & 0.35\\
150 & 0.24 & 0.31 & 0.44 & 0.61 & 0.81 & 0.27\\
155 & 0.23 & 0.31 & 0.43 & 0.61 & 0.81 & 0.27\\
160 & 0.26 & 0.35 & 0.49 & 0.69 & 0.94 & 0.33\\
\hline
\end{tabular}
\end{center}
\end{table}

\begin{table}
\caption{95\% \CL exclusion limits in \% for the \ljets from combined \PQc tagging categories.}
\label{tab:limit_lepton}
\begin{center}
\begin{tabular}{ccccccc}
\hline
\hline
\multicolumn{1}{c}{$\mHp$ (GeV)} & \multicolumn{5}{c}{{Expected}} & \multicolumn{1}{c}{{Observed}} \\
&\multicolumn{1}{c}{$-2\sigma$} &\multicolumn{1}{c}{$-1\sigma$} &\multicolumn{1}{c}{median} & \multicolumn{1}{c}{$+1\sigma$} & \multicolumn{1}{c}{$+2\sigma$}&\\ \hline
\hline
80  & 1.33 & 1.77 & 2.46 & 3.45 & 4.63 & 1.65\\
90  & 0.56 & 0.74 & 1.03 & 1.43 & 1.90 & 0.68\\
100 & 0.29 & 0.38 & 0.53 & 0.73 & 0.98 & 0.30\\
120 & 0.20 & 0.26 & 0.36 & 0.51 & 0.67 & 0.25\\
140 & 0.18 & 0.23 & 0.32 & 0.45 & 0.59 & 0.21\\
150 & 0.17 & 0.22 & 0.31 & 0.43 & 0.57 & 0.20\\
155 & 0.17 & 0.22 & 0.31 & 0.44 & 0.58 & 0.22\\
160 & 0.18 & 0.24 & 0.34 & 0.47 & 0.64 & 0.25\\
\hline
\end{tabular}
\end{center}
\end{table}
